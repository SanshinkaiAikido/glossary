\def\gryu{{
\noindent\par\begin{tabular}{@{}p{2.75cm}p{5cm}p{8.25cm}@{}}
{\fontspec{Sazanami Gothic}\Large\bfseries 竜}	&\Large{\bfseries{ry\=u}}	& \Large{Type}\\
	&& \\
{\fontspec{Sazanami Gothic}合気道}	&ai·ki·d\=o	& Weg des harmoniseren von Ki\\
{\fontspec{Sazanami Gothic}合気?}	&ai·ki·jutsu	& Korpertechniken die aiki Prinzipien verwenden\\
{\fontspec{Sazanami Gothic}合気杖}	&ai·ki·j\=o	& j\=o Techniken die aiki Prinzipien verwenden\\
{\fontspec{Sazanami Gothic}合気剣}	&ai·ki·ken	& ken Techniken die aiki Prinzipien verwenden\\
{\fontspec{Sazanami Gothic}法定形}	&ho·jo kata	& ersten Kata der spezifischen Schwert-Stil\\
{\fontspec{Sazanami Gothic}鹿島神傳直心影流}	&Kashima Shinden Jikishinkage-ry\=u	& Schwertstil\\
{\fontspec{Sazanami Gothic}元気会}	&gen·ki·kai	& Gruppe der gesundheits Übungen\\
{\fontspec{Sazanami Gothic}整体}	&sei·tai	& ?\\
\end{tabular}
\vspace{.5cm}
}}

\def\gkeikoho{{
\noindent\par\begin{tabular}{@{}p{2.75cm}p{5cm}p{8.25cm}@{}}
{\fontspec{Sazanami Gothic}\Large\bfseries ?}	&\Large{\bfseries{keiko h\=o}}	& \Large{TODOGeist}\\
	&& \\
{\fontspec{Sazanami Gothic}?稽古}	&kakari kei·ko	& angreifen einer nach dem anderen [fortwährend üben]\\
{\fontspec{Sazanami Gothic}?}	&kigata	& fliessende form, tori lässt sich nicht festhalten [energetische form]\\
{\fontspec{Sazanami Gothic}鍛錬}	&tan·ren	& dizipliniertes training, tori lässt sich fessthalten [diszipliniert]\\
{\fontspec{Sazanami Gothic}真剣}	&shin·ken	& realistisches training [ernst]\\
{\fontspec{Sazanami Gothic}崩し}	&kuzushi	& das Gleichgewicht brechen\\
{\fontspec{Sazanami Gothic}後ろ技}	&ushi·ro·waza	& techniken bei denen man von hinten angegriffen wird\\
{\fontspec{Sazanami Gothic}自由技}	&ji·y\=u waza	& freie Technik\\
{\fontspec{Sazanami Gothic}乱取り}	&ran·do·ri	& techniken mit mehreren Angreifer\\
{\fontspec{Sazanami Gothic}素振り}	&su·bu·ri	&  [?]\\
\end{tabular}
\vspace{.5cm}
}}

\def\g?{{
\noindent\par\begin{tabular}{@{}p{2.75cm}p{5cm}p{8.25cm}@{}}
{\fontspec{Sazanami Gothic}\Large\bfseries ?}	&\Large{\bfseries{?}}	& \Large{Position}\\
	&& \\
{\fontspec{Sazanami Gothic}座り技}	&suwa·ri waza	& sitzende Technik\\
{\fontspec{Sazanami Gothic}半身半立ち技}	&han·mi han·da·chi waza	& Technik für sitzender tori und stehender uke\\
{\fontspec{Sazanami Gothic}立ち技}	&ta·chi waza	& stehende Technik\\
{\fontspec{Sazanami Gothic}座}	&za	& sitzend\\
{\fontspec{Sazanami Gothic}正座}	&sei·za	& auf der Knien sitzend mit Füßen flach auf dem Boden\\
{\fontspec{Sazanami Gothic}跪座}	&ki·za	& auf der Knien sitzend mit Zehen in dem Boden\\
{\fontspec{Sazanami Gothic}割座}	&wari·za	& auf den Knien sitzend mit Hüften zwischen dem Füßen auf dem Boden\\
{\fontspec{Sazanami Gothic}?座}	&cho·za	& sitzend mit Beine vorausgestreckt\\
{\fontspec{Sazanami Gothic}?座}	&kai·za	& sitzend mit Beine gestreckt und geöffnet\\
{\fontspec{Sazanami Gothic}体座}	&tai·za	& sitzend mit Knien gegen die Brust und Füßen auf dem Boden\\
{\fontspec{Sazanami Gothic}?座}	&gaku·za	& sitzend mit Fußsohlen gegeneinander und Knien auf dem Bodem\\
{\fontspec{Sazanami Gothic}安座}	&an·za	& Schneidersitz ohne die Beinen zu kreuzen\\
{\fontspec{Sazanami Gothic}八相(八双)}	&hass\=o	& Haltung wie eine Acht [hass\=ogamae]\\
{\fontspec{Sazanami Gothic}?座}	&kahuza	& Lotusposition\\
\end{tabular}
\vspace{.5cm}
}}

\def\gaite{{
\noindent\par\begin{tabular}{@{}p{2.75cm}p{5cm}p{8.25cm}@{}}
{\fontspec{Sazanami Gothic}\Large\bfseries 相手}	&\Large{\bfseries{ai·te}}	& \Large{Partner}\\
	&& \\
{\fontspec{Sazanami Gothic}多人数取り}	&ta·nin·z\=u to·ri	& Drei oder mehr uke die fortlaufend angreifen, mache nur begin der Techik\\
{\fontspec{Sazanami Gothic}掛り稽古}	&ka·ka·ri gei·ko	& zwei oder mehr uke die abwechselend angreifen\\
\end{tabular}
\vspace{.5cm}
}}

\def\gbuki{{
\noindent\par\begin{tabular}{@{}p{2.75cm}p{5cm}p{8.25cm}@{}}
{\fontspec{Sazanami Gothic}\Large\bfseries 武器}	&\Large{\bfseries{bu·ki}}	& \Large{Waffe}\\
	&& \\
{\fontspec{Sazanami Gothic}杖}	&j\=o	& Stock, kurzer holzen Stab\\
{\fontspec{Sazanami Gothic}棒}	&b\=o	& langer Holzstab\\
{\fontspec{Sazanami Gothic}木剣}	&bok·ken, boku·t\=o	& gebogenes Holzschwert\\
{\fontspec{Sazanami Gothic}短刀}	&tan·t\=o	& Messer oder Dolch\\
{\fontspec{Sazanami Gothic}脇差}	&waki·zashi, shoto	& gebogenes Kurzschwert aus Metall\\
{\fontspec{Sazanami Gothic}刀}	&(ken?,) katana	& gebogenes Metallschwert\\
{\fontspec{Sazanami Gothic}太刀}	&ta·chi	& Langschwert aus Metall\\
{\fontspec{Sazanami Gothic}剣}	&ken	& kurzes, grades zweischnediges Metallschwert\\
{\fontspec{Sazanami Gothic}日本刀}	&ni·hon·t\=o	& Japanische Schwert\\
{\fontspec{Sazanami Gothic}薙刀}	&naginata	& Hellebarde\\
{\fontspec{Sazanami Gothic}槍}	&yari	& Speer\\
{\fontspec{Sazanami Gothic}鍔}	&tsuba	& ?\\
{\fontspec{Sazanami Gothic}鞘}	&saya	& Scheide\\
{\fontspec{Sazanami Gothic}?}	&bugukake	& Waffenreck?\\
\end{tabular}
\vspace{.5cm}
}}

\def\gkokukegi{{
\noindent\par\begin{tabular}{@{}p{2.75cm}p{5cm}p{8.25cm}@{}}
{\fontspec{Sazanami Gothic}\Large\bfseries 迫撃}	&\Large{\bfseries{koku·kegi}}	& \Large{Angriff}\\
	&& \\
{\fontspec{Sazanami Gothic}気合せ正面打ち}	&ki·a·wa·se sh\=o·men u·chi	& Stoß mit geöffneter Hand von den Hüften nach vorne zum Kopf\\
{\fontspec{Sazanami Gothic}片手取り相半身}	&kata·te do·ri ai han·mi	& eine Hand fasst das selbe Handgelenk des Gegenübers (identisch) [eine Hand fasst die selbe Hand des Gegenübers (identisch)]\\
{\fontspec{Sazanami Gothic}片手取り逆半身}	&kata·te do·ri gyaku han·mi	& eine Hand fasst das andere Handgelenk des Gegenübers (spiegel) [eine Hand fasst das andere Handgelenk des Gegenübers (spiegel)]\\
{\fontspec{Sazanami Gothic}両手取り}	&ry\=o·te do·ri	& beide Handgelenke fassen [beide Handgelenke fassen]\\
{\fontspec{Sazanami Gothic}正面打ち}	&sh\=o·men u·chi	& Schlag gegen Vorne von Kopf [fracesco]\\
{\fontspec{Sazanami Gothic}片手両手取り}	&kata·te ry\=o·te do·ri	& ?\\
{\fontspec{Sazanami Gothic}肩取り}	&kata do·ri	& die Schulter fassen\\
{\fontspec{Sazanami Gothic}後ろ両手取り}	&ushi·ro ry\=o·te do·ri	& dem Partner von hinten an die Handgelenke fassen\\
{\fontspec{Sazanami Gothic}横面打ち}	&yoko·men u·chi	& Schlag gegen die Schläfe\\
{\fontspec{Sazanami Gothic}中段突き}	&ch\=u·dan tsu·ki	& Stoß nach dem Mittelteil des Körpers\\
{\fontspec{Sazanami Gothic}肩取り面打ち}	&kata do·ri men u·chi	& Schulter fassen und Schlag zum Gesicht\\
{\fontspec{Sazanami Gothic}後ろ両肩取り}	&ushi·ro ry\=o·kata do·ri	& von Hinten beide Schulter fassen\\
{\fontspec{Sazanami Gothic}上段突き}	&j\=o·dan tsu·ki	& Stoß nach dem oberen Teil des Körpers\\
{\fontspec{Sazanami Gothic}両襟取り}	&ry\=o·eri do·ri	& ssk newsletter 3XXX\\
{\fontspec{Sazanami Gothic}蹴り}	&ke·ri	& Tritt\\
{\fontspec{Sazanami Gothic}後ろ首絞め}	&ushi·ro kubi jime	& Würgen von hinten\\
\end{tabular}
\vspace{.5cm}
}}

\def\guchi/soto{{
\noindent\par\begin{tabular}{@{}p{2.75cm}p{5cm}p{8.25cm}@{}}
{\fontspec{Sazanami Gothic}\Large\bfseries 内/外}	&\Large{\bfseries{uchi/soto}}	& \Large{innen/ausen}\\
	&& \\
{\fontspec{Sazanami Gothic}内捌き}	&uchi saba·ki	& Bewegung nach innen\\
{\fontspec{Sazanami Gothic}外捌き}	&soto saba·ki	& Bewegung nach draussen\\
\end{tabular}
\vspace{.5cm}
}}

\def\gtaisabaki{{
\noindent\par\begin{tabular}{@{}p{2.75cm}p{5cm}p{8.25cm}@{}}
{\fontspec{Sazanami Gothic}\Large\bfseries 体捌き}	&\Large{\bfseries{tai saba·ki}}	& \Large{Korperbewegung}\\
	&& \\
{\fontspec{Sazanami Gothic}前足入身}	&mae ashi iri·mi	& eintreten mit vorderem Fuß\\
{\fontspec{Sazanami Gothic}後ろ足入身}	&ushi·ro ashi iri·mi	& eintreten mit hinterem Fuß\\
{\fontspec{Sazanami Gothic}前足転換}	&mae ashi ten·kan	& wegdrehen auf vorderem Fuß\\
{\fontspec{Sazanami Gothic}後ろ足転換}	&ushi·ro ashi ten·kan	& wegdrehen auf hinterem Fuß\\
{\fontspec{Sazanami Gothic}前足?}	&mae ashi tenshin	& ausweichen mit vorderem Fuß\\
{\fontspec{Sazanami Gothic}後ろ足?}	&ushi·ro ashi tenshin	& ausweichen mit hinterem Fuß\\
{\fontspec{Sazanami Gothic}前足入身転換}	&mae ashi iri·mi ten·kan	& eintreten mit Vorderfuss mit anschliessender Drehung\\
{\fontspec{Sazanami Gothic}後ろ足入身転換}	&ushi·ro ashi iri·mi ten·kan	& eintreten mit Hinterfuss mit anschliessender Drehung\\
{\fontspec{Sazanami Gothic}前足入身転換?}	&mae ashi iri·mi ten·kan tenshin	& eintreten mit Vorderfuss,Körperdrehung und ausweichen\\
{\fontspec{Sazanami Gothic}後ろ足入身転換?}	&ushi·ro ashi iri·mi ten·kan tenshin	& eintreten mit Hinterfuss,Körperdrehung und ausweichen\\
{\fontspec{Sazanami Gothic}前足転換?}	&mae ashi ten·kan tenshin	& Körperdrehung mit dem Vorderfuss und ausweichen\\
{\fontspec{Sazanami Gothic}後ろ足転換?}	&ushi·ro ashi ten·kan tenshin	& Körperdrehung mit dem Hinterfuss und ausweichen\\
{\fontspec{Sazanami Gothic}後ろ足入身?}	&ushi·ro ashi iri·mi tenkai	& eintreten mit Hinterfuss und anschliessender Körperdrehung\\
\end{tabular}
\vspace{.5cm}
}}

\def\gtesabaki{{
\noindent\par\begin{tabular}{@{}p{2.75cm}p{5cm}p{8.25cm}@{}}
{\fontspec{Sazanami Gothic}\Large\bfseries 手捌き}	&\Large{\bfseries{te saba·ki}}	& \Large{Handbewegung}\\
	&& \\
{\fontspec{Sazanami Gothic}上半円}	&kami han·en	& halber Kreis gegen oben\\
{\fontspec{Sazanami Gothic}下半円}	&shimo han·en	& Halbkreis gegen unten\\
{\fontspec{Sazanami Gothic}手首返し}	&te·kubi gae·shi	& ?\\
{\fontspec{Sazanami Gothic}十字結び}	&jy\=u·ji musu·bi	& ?\\
{\fontspec{Sazanami Gothic}受け流し}	&uke naga·shi	& ?\\
\end{tabular}
\vspace{.5cm}
}}

\def\gnote{{
\noindent\par\begin{tabular}{@{}p{2.75cm}p{5cm}p{8.25cm}@{}}
{\fontspec{Sazanami Gothic}\Large\bfseries の手}	&\Large{\bfseries{no·te}}	& \Large{Hand}\\
	&& \\
{\fontspec{Sazanami Gothic}打ちの手}	&u·chi·no·te	& Schlaghand\\
{\fontspec{Sazanami Gothic}肩の手}	&kata·no·te	& Hand bei Schulter\\
\end{tabular}
\vspace{.5cm}
}}

\def\gdan{{
\noindent\par\begin{tabular}{@{}p{2.75cm}p{5cm}p{8.25cm}@{}}
{\fontspec{Sazanami Gothic}\Large\bfseries 段}	&\Large{\bfseries{dan}}	& \Large{Höhe}\\
	&& \\
{\fontspec{Sazanami Gothic}下段}	&ge·dan	& Länge\\
{\fontspec{Sazanami Gothic}中段}	&ch\=u·dan	& ?\\
{\fontspec{Sazanami Gothic}上段}	&j\=o·dan	& hoch\\
{\fontspec{Sazanami Gothic}脇構}	&waki gamae	& schmale Haltung\\
{\fontspec{Sazanami Gothic}下段の構え}	&ge·dan no kama·e	& tiefe Haltung\\
{\fontspec{Sazanami Gothic}中段の構え}	&ch\=u·dan no kama·e	& mittlere Haltung\\
{\fontspec{Sazanami Gothic}上段の構}	&j\=o·dan no kamae	& hohe Haltung\\
\end{tabular}
\vspace{.5cm}
}}

\def\gwaza{{
\noindent\par\begin{tabular}{@{}p{2.75cm}p{5cm}p{8.25cm}@{}}
{\fontspec{Sazanami Gothic}\Large\bfseries 技}	&\Large{\bfseries{waza}}	& \Large{Technik}\\
	&& \\
{\fontspec{Sazanami Gothic}一教}	&ik·ky\=o	&  [Lektion Nummer eins]\\
{\fontspec{Sazanami Gothic}二教}	&ni·ky\=o	&  [Lektion Nummer zwei]\\
{\fontspec{Sazanami Gothic}三教}	&san·ky\=o	&  [Lektion Nummer drei]\\
{\fontspec{Sazanami Gothic}四教}	&yon·ky\=o	&  [Lektion Nummer vier]\\
{\fontspec{Sazanami Gothic}隅落し}	&sumi·oto·shi	& ?\\
{\fontspec{Sazanami Gothic}小手返し}	&ko·te·gae·shi	& Handgelenk verdrehen\\
{\fontspec{Sazanami Gothic}入身投げ}	&iri·mi·na·ge	& ?\\
{\fontspec{Sazanami Gothic}四方投げ}	&shi·h\=o·na·ge	& schneiden in vier Richtungen\\
{\fontspec{Sazanami Gothic}五教}	&go·ky\=o	& Lektion Nummer fünf\\
{\fontspec{Sazanami Gothic}肘決め抑え}	&hiji ki·me osa·e	& Ellbogen controllierende blockieren\\
{\fontspec{Sazanami Gothic}内回転三教}	&uchi·kai·ten·san·ky\=o	& ?\\
{\fontspec{Sazanami Gothic}腕絡み}	&ude gara·mi	& ?\\
{\fontspec{Sazanami Gothic}合気腰}	&ai·ki·goshi	& ?\\
{\fontspec{Sazanami Gothic}合気落し}	&ai·ki·oto·shi	&  [?]\\
{\fontspec{Sazanami Gothic}回転投げ}	&kai·ten·na·ge	&  [?]\\
{\fontspec{Sazanami Gothic}腕決め投げ}	&ude ki·me na·ge	&  [?]\\
{\fontspec{Sazanami Gothic}前落し}	&mae·oto·shi	&  [?]\\
{\fontspec{Sazanami Gothic}引き落し}	&hiki·oto·shi	&  [?]\\
{\fontspec{Sazanami Gothic}切り落し}	&ki·ri oto·shi	&  [Wurf mit Schnitt]\\
{\fontspec{Sazanami Gothic}回転落し}	&kai·ten·oto·shi	& Drehwurf\\
{\fontspec{Sazanami Gothic}天地投げ}	&ten chi na·ge	& Himmel-Erde Wurf\\
{\fontspec{Sazanami Gothic}玄形呼吸投げ}	&gen·kei·ko·ky\=u·na·ge	& ?\\
{\fontspec{Sazanami Gothic}内回転投げ}	&uchi·kai·ten·na·ge	& ?\\
{\fontspec{Sazanami Gothic}振り突き呼吸投げ}	&furi·zu·ki ko·ky\=u na·ge	& ?\\
{\fontspec{Sazanami Gothic}十字絡み}	&jy\=u·ji gara·mi	&  [?]\\
{\fontspec{Sazanami Gothic}鳥り船呼吸投げ}	&to·ri fune ko·ky\=u·na·ge	&  [?]\\
{\fontspec{Sazanami Gothic}四方蹴り呼吸投げ}	&shi·h\=o ge·ri ko·ky\=u·na·ge	&  [?]\\
{\fontspec{Sazanami Gothic}教絡み三教投げ}	&ude gara·mi san·ky\=o na·ge	& ?\\
{\fontspec{Sazanami Gothic}教絡み四教投げ}	&ude gara·mi yon·ky\=o na·ge	& ?\\
{\fontspec{Sazanami Gothic}腰投げ}	&koshi na·ge	& Hüftwurf\\
{\fontspec{Sazanami Gothic}外回転投げ}	&soto kai·ten·na·ge	& ?\\
{\fontspec{Sazanami Gothic}斬刀呼吸投げ}	&zan·to ko·ky\=u na·ge	& ?\\
{\fontspec{Sazanami Gothic}教絡み抑え}	&ude gara·mi osa·e	& ?\\
{\fontspec{Sazanami Gothic}手車}	&te guruma	& ?\\
{\fontspec{Sazanami Gothic}背負い車}	&se·o·i guruma	& ?\\
{\fontspec{Sazanami Gothic}腰車}	&koshi guruma	& ?\\
{\fontspec{Sazanami Gothic}沈身腰車}	&chin shin koshi guruma	& ?\\
\end{tabular}
\vspace{.5cm}
}}

\def\gomote/ura{{
\noindent\par\begin{tabular}{@{}p{2.75cm}p{5cm}p{8.25cm}@{}}
{\fontspec{Sazanami Gothic}\Large\bfseries 表/裏}	&\Large{\bfseries{omote/ura}}	& \Large{omote/ura}\\
	&& \\
{\fontspec{Sazanami Gothic}表}	&omote	& ?\\
{\fontspec{Sazanami Gothic}裏}	&ura	& ?\\
\end{tabular}
\vspace{.5cm}
}}

\def\gyin/yang{{
\noindent\par\begin{tabular}{@{}p{2.75cm}p{5cm}p{8.25cm}@{}}
{\fontspec{Sazanami Gothic}\Large\bfseries ?/?}	&\Large{\bfseries{yin/yang}}	& \Large{Yin/Yang}\\
	&& \\
{\fontspec{Sazanami Gothic}?}	&yin	& Yin\\
{\fontspec{Sazanami Gothic}?}	&yang	& Yang\\
\end{tabular}
\vspace{.5cm}
}}

\def\gri{{
\noindent\par\begin{tabular}{@{}p{2.75cm}p{5cm}p{8.25cm}@{}}
{\fontspec{Sazanami Gothic}\Large\bfseries 理}	&\Large{\bfseries{ri}}	& \Large{Prinzip}\\
	&& \\
{\fontspec{Sazanami Gothic}水}	&su, mizu	& Wasser, oben-unten, Ost\\
{\fontspec{Sazanami Gothic}土}	&do, tsu	& Erde, links-rechts, Süd\\
{\fontspec{Sazanami Gothic}風}	&hu	& LuftXXX, vorne-hinten, West\\
{\fontspec{Sazanami Gothic}火}	&ka, hi	& Feuer, Spirale, Nord\\
{\fontspec{Sazanami Gothic}?}	&complete?	& Mensch\\
{\fontspec{Sazanami Gothic}春}	&haru	& Frühling\\
{\fontspec{Sazanami Gothic}夏}	&natsu	& Sommer\\
{\fontspec{Sazanami Gothic}秋}	&aki	& Herbst\\
{\fontspec{Sazanami Gothic}冬}	&fuyu	& Winter\\
{\fontspec{Sazanami Gothic}攻防の原理}	&k\=o·b\=o no gen·ri	& ?\\
{\fontspec{Sazanami Gothic}打ちの理}	&u·chi no ri	& Prinzip des schlagens\\
{\fontspec{Sazanami Gothic}抑えの理}	&osa·e no ri	& ?\\
{\fontspec{Sazanami Gothic}投げの理}	&na·ge no ri	& Prinzip des Wurfes\\
{\fontspec{Sazanami Gothic}斬の理}	&zan no ri	& ?\\
\end{tabular}
\vspace{.5cm}
}}

\def\gnage/osae{{
\noindent\par\begin{tabular}{@{}p{2.75cm}p{5cm}p{8.25cm}@{}}
{\fontspec{Sazanami Gothic}\Large\bfseries 投げ/抑え}	&\Large{\bfseries{nage/osae}}	& \Large{nage/osae}\\
	&& \\
{\fontspec{Sazanami Gothic}投げ}	&na·ge	& Wurf\\
{\fontspec{Sazanami Gothic}抑え}	&osa·e	& festhalten\\
{\fontspec{Sazanami Gothic}投げ抑え}	&na·ge osa·e	& Wurf festhalten\\
\end{tabular}
\vspace{.5cm}
}}

\def\gukemi{{
\noindent\par\begin{tabular}{@{}p{2.75cm}p{5cm}p{8.25cm}@{}}
{\fontspec{Sazanami Gothic}\Large\bfseries 受身}	&\Large{\bfseries{u·ke·mi}}	& \Large{empfangen mit dem Körper}\\
	&& \\
	&boven onder u·ke·mi	& ?\\
{\fontspec{Sazanami Gothic}前受身}	&mae u·ke·mi	& forwärts rollen\\
{\fontspec{Sazanami Gothic}後ろ受身}	&ushi·ro u·ke·mi	& nach Hinten rollen\\
{\fontspec{Sazanami Gothic}横受身}	&yoko u·ke·mi	& seitwärts ukemi vom links nach rechts(Prinzip der Spyrale)\\
	&chokuto	& ?\\
{\fontspec{Sazanami Gothic}飛び受身}	&to·bi u·ke·mi	& freie Fall\\
\end{tabular}
\vspace{.5cm}
}}

\def\g?{{
\noindent\par\begin{tabular}{@{}p{2.75cm}p{5cm}p{8.25cm}@{}}
{\fontspec{Sazanami Gothic}\Large\bfseries ?}	&\Large{\bfseries{?}}	& \Large{Person}\\
	&& \\
{\fontspec{Sazanami Gothic}植芝 盛平}	&Ueshiba, Morihei	& ehemaliger Gründer von aikid\=o, \=osensei (14-12-1883 - 26-04-1969)\\
{\fontspec{Sazanami Gothic}植芝 吉祥丸}	&Ueshiba, Kisshomaru	& ehemaliger Sohn von \=osensei, zweite d\=oshu (27-06-1921 - 04-01-1999)\\
{\fontspec{Sazanami Gothic}植芝 守央}	&Moriteru Ueshiba	& Enkelkind von \=osensei, dritte d\=oshu (02-04-1951)\\
{\fontspec{Sazanami Gothic}多田 宏}	&Tada, Hiroshi	& Student von \=osensei, shihan (13-12-1929)\\
{\fontspec{Sazanami Gothic}池田 昌富}	&Ikeda, Masatomi	& Student von Tada sensei, shihan (08-04-1940)\\
\end{tabular}
\vspace{.5cm}
}}

\def\gkazueru{{
\noindent\par\begin{tabular}{@{}p{2.75cm}p{5cm}p{8.25cm}@{}}
{\fontspec{Sazanami Gothic}\Large\bfseries 数える}	&\Large{\bfseries{kazu·e·ru}}	& \Large{zehlen}\\
	&& \\
{\fontspec{Sazanami Gothic}○}	&zero	& null, leer\\
{\fontspec{Sazanami Gothic}一}	&ichi	& eins\\
{\fontspec{Sazanami Gothic}二}	&ni	& zwei\\
{\fontspec{Sazanami Gothic}三}	&san	& drei\\
{\fontspec{Sazanami Gothic}四}	&shi, yon	& vier\\
{\fontspec{Sazanami Gothic}五}	&go	& fünf\\
{\fontspec{Sazanami Gothic}六}	&roku	& sechs\\
{\fontspec{Sazanami Gothic}七}	&shichi, nana	& sieben\\
{\fontspec{Sazanami Gothic}八}	&hachi	& acht\\
{\fontspec{Sazanami Gothic}九}	&ku, ky\=u	& neun\\
{\fontspec{Sazanami Gothic}十}	&j\=u	& zehn\\
{\fontspec{Sazanami Gothic}十一}	&j\=u ichi	& elf\\
{\fontspec{Sazanami Gothic}二十}	&ni j\=u	& zwanzig\\
{\fontspec{Sazanami Gothic}二十一}	&ni j\=u ichi	& einundzwanzig\\
{\fontspec{Sazanami Gothic}百}	&hyaku	& hundert\\
{\fontspec{Sazanami Gothic}千}	&sen	& tausent\\
{\fontspec{Sazanami Gothic}万}	&man	& zehntausent\\
{\fontspec{Sazanami Gothic}一本目}	&ip·pon·me	& erste Teil\\
{\fontspec{Sazanami Gothic}ニ本目}	&ni·hon·me	& tweite Teil\\
{\fontspec{Sazanami Gothic}三本目}	&san·bon·me	& dritte Teil\\
{\fontspec{Sazanami Gothic}四本目}	&yon·hon·me	& fierte Teil\\
{\fontspec{Sazanami Gothic}五本目}	&go·hon·me	& fünfte Teil\\
{\fontspec{Sazanami Gothic}六本目}	&roku·hon·me	& zechste Teil\\
{\fontspec{Sazanami Gothic}七本目}	&nana·hon·me	& siebenste Teil\\
{\fontspec{Sazanami Gothic}八本目}	&hachi·hon·me	& achtste Teil\\
\end{tabular}
\vspace{.5cm}
}}

\def\gdankai{{
\noindent\par\begin{tabular}{@{}p{2.75cm}p{5cm}p{8.25cm}@{}}
{\fontspec{Sazanami Gothic}\Large\bfseries 段階}	&\Large{\bfseries{dan·kai}}	& \Large{Graden}\\
	&& \\
{\fontspec{Sazanami Gothic}段}	&dan	& dan grad\\
{\fontspec{Sazanami Gothic}級}	&ky\=u	& ky\=u ?\\
{\fontspec{Sazanami Gothic}無級}	&mu·ky\=u	& ohne ky\=u\\
{\fontspec{Sazanami Gothic}六級}	&rok·ky\=u	& segste ky\=u\\
{\fontspec{Sazanami Gothic}五級}	&go·ky\=u	& fünfte ky\=u\\
{\fontspec{Sazanami Gothic}四級}	&yon·ky\=u	& vierte ky\=u\\
{\fontspec{Sazanami Gothic}参級}	&san·ky\=u	& dritten ky\=u\\
{\fontspec{Sazanami Gothic}弐級}	&ni·ky\=u	& zweiter ky\=u\\
{\fontspec{Sazanami Gothic}壱級}	&ik·ky\=u	& erste ky\=u, Anfengerky\=u\\
{\fontspec{Sazanami Gothic}有段者}	&y\=u·dan·sha	& mit Dan\\
{\fontspec{Sazanami Gothic}無段者}	&mu·dan·sha	& ohne Dan\\
{\fontspec{Sazanami Gothic}初段}	&sho·dan	& erste Dan, Anfengerdan\\
{\fontspec{Sazanami Gothic}弐段}	&ni·dan	& zweiter Dan\\
{\fontspec{Sazanami Gothic}参段}	&san·dan	& dritten Dan\\
{\fontspec{Sazanami Gothic}四段}	&yon·dan	& vierte Dan\\
{\fontspec{Sazanami Gothic}五段}	&go·dan	& funfte Dan\\
{\fontspec{Sazanami Gothic}六段}	&roku·dan	& segste Dan\\
{\fontspec{Sazanami Gothic}七段}	&nana·dan	& sieben Dan\\
{\fontspec{Sazanami Gothic}八段}	&hachi·dan	& achtste Dan\\
{\fontspec{Sazanami Gothic}九段}	&ku·dan	& neunte Dan\\
{\fontspec{Sazanami Gothic}十段}	&jy\=u·dan	& zhente Dan\\
\end{tabular}
\vspace{.5cm}
}}

\def\g?{{
\noindent\par\begin{tabular}{@{}p{2.75cm}p{5cm}p{8.25cm}@{}}
{\fontspec{Sazanami Gothic}\Large\bfseries ?}	&\Large{\bfseries{?}}	& \Large{Algemein}\\
	&& \\
{\fontspec{Sazanami Gothic}合}	&ai	& Harmonie, vereinend\\
{\fontspec{Sazanami Gothic}気}	&ki	& Lebenskraft, Lebensenergie\\
{\fontspec{Sazanami Gothic}道}	&d\=o	& Weg, ?, Disciplin\\
{\fontspec{Sazanami Gothic}合気道}	&ai·ki·d\=o	& Weg des harmoniseren von Ki\\
{\fontspec{Sazanami Gothic}合気道家}	&ai·ki·d\=o·ka	& Eingeweihter\\
{\fontspec{Sazanami Gothic}合気会}	&ai·ki·kai	& Organisation fur die ubing und ? von aikid\=o\\
{\fontspec{Sazanami Gothic}受け}	&u·ke	& Angreifer [entfanger]\\
{\fontspec{Sazanami Gothic}取り}	&to·ri	& Verteitiger [?]\\
{\fontspec{Sazanami Gothic}投げ}	&na·ge	&  [?]\\
{\fontspec{Sazanami Gothic}打太刀}	&uchi·da·chi	& Elter, Lehrer [?]\\
{\fontspec{Sazanami Gothic}受太刀}	&shi·da·chi	& Kind, Schuler [?]\\
\end{tabular}
\vspace{.5cm}
}}

\def\g?{{
\noindent\par\begin{tabular}{@{}p{2.75cm}p{5cm}p{8.25cm}@{}}
{\fontspec{Sazanami Gothic}\Large\bfseries ?}	&\Large{\bfseries{?}}	& \Large{Instruktionen}\\
	&& \\
{\fontspec{Sazanami Gothic}お願いします}	&o·nega·i shi·ma·su	& bitte (am Anfang vom Trainung oder einer Übung)\\
{\fontspec{Sazanami Gothic}どうも有難う御座いました}	&do·o·mo a·ri·ga·to·o go·za·i·ma·shi·ta	& danke (am Ende vom Training oder einer Übung)\\
{\fontspec{Sazanami Gothic}はい}	&hai	& ja\\
{\fontspec{Sazanami Gothic}止め}	&yame	& halt (und setzt Euch auf die Seite)\\
{\fontspec{Sazanami Gothic}始め}	&haji·me	& fangt an, startet\\
{\fontspec{Sazanami Gothic}待って}	&ma·t·te	& wartet, stopt\\
{\fontspec{Sazanami Gothic}立て}	&ta·te	& steht auf\\
{\fontspec{Sazanami Gothic}?}	&suwatte	& setz Euch\\
{\fontspec{Sazanami Gothic}?}	&mawatte	& dreht Euch\\
{\fontspec{Sazanami Gothic}個体}	&ko·tai	& Partnerwechseln\\
{\fontspec{Sazanami Gothic}?理解}	&mo·ri·kai	& wiederholt die Übung\\
{\fontspec{Sazanami Gothic}反対}	&han·tai	& braucht die andere Seite\\
{\fontspec{Sazanami Gothic}相手?}	&ai·te kaite	& finde einen anderen Partner\\
{\fontspec{Sazanami Gothic}礼}	&rei	& grüssen [Respect]\\
{\fontspec{Sazanami Gothic}?礼}	&ritsu·rei	& stehend grüssen\\
{\fontspec{Sazanami Gothic}座礼}	&za·rei	& sitzend grüssen\\
\end{tabular}
\vspace{.5cm}
}}

\def\g?{{
\noindent\par\begin{tabular}{@{}p{2.75cm}p{5cm}p{8.25cm}@{}}
{\fontspec{Sazanami Gothic}\Large\bfseries ?}	&\Large{\bfseries{?}}	& \Large{Titels}\\
	&& \\
{\fontspec{Sazanami Gothic}大先生}	&\=o·sen·sei	& Grossmeister Lehrer\\
{\fontspec{Sazanami Gothic}先生}	&sen·sei	& Lehrer, Instructor\\
{\fontspec{Sazanami Gothic}道主}	&d\=o·shu	& der jenige der den Weg weist\\
{\fontspec{Sazanami Gothic}師範}	&shi·han	& Haubtinstruktor (minimal 6e dan)\\
{\fontspec{Sazanami Gothic}指導員}	&shi·do·in	& Instructor, shihan-in-Ausbildung (4e oder 5e dan)\\
{\fontspec{Sazanami Gothic}副指導員}	&fuku·shi·do·in	& assistent Instructor (2e oder 3e dan)\\
{\fontspec{Sazanami Gothic}先輩}	&sen·pai	& jemands Senior\\
{\fontspec{Sazanami Gothic}後輩}	&k\=o·hai	& jemands Junior\\
{\fontspec{Sazanami Gothic}無段者}	&mu·dan·sha	& Person ohne Grad\\
{\fontspec{Sazanami Gothic}有段者}	&y\=u·dan·sha	& Person mit Grad\\
{\fontspec{Sazanami Gothic}道場長}	&d\=o·j\=o·cho	& d\=oj\=o-Haubt\\
{\fontspec{Sazanami Gothic}弟子}	&de·shi	& Student\\
{\fontspec{Sazanami Gothic}内弟子}	&uchi·de·shi	& im d\=oj\=o lebender Student\\
{\fontspec{Sazanami Gothic}外弟子}	&soto·de·shi	& auserhald dem d\=oj\=o lebender Student\\
\end{tabular}
\vspace{.5cm}
}}

\def\g?{{
\noindent\par\begin{tabular}{@{}p{2.75cm}p{5cm}p{8.25cm}@{}}
{\fontspec{Sazanami Gothic}\Large\bfseries ?}	&\Large{\bfseries{?}}	& \Large{Objecten}\\
	&& \\
{\fontspec{Sazanami Gothic}道場}	&d\=o·j\=o	& Trennungsraum (Platz von Erleuchtung) [buchstäblich Platz von dem Weg]\\
{\fontspec{Sazanami Gothic}上座}	&kamiza	& Altar mit Portret von \=osensei (eventuel mit Kalligrafie oder Blume)\\
{\fontspec{Sazanami Gothic}?座}	&shomiza	&  [?]\\
{\fontspec{Sazanami Gothic}?}	&joseki	& ?\\
{\fontspec{Sazanami Gothic}?}	&shimoseki	& ?\\
{\fontspec{Sazanami Gothic}畳}	&tatami	& Matte\\
{\fontspec{Sazanami Gothic}本部}	&hom·bu	& Hauptkwartier\\
\end{tabular}
\vspace{.5cm}
}}

\def\gifuku{{
\noindent\par\begin{tabular}{@{}p{2.75cm}p{5cm}p{8.25cm}@{}}
{\fontspec{Sazanami Gothic}\Large\bfseries 衣服}	&\Large{\bfseries{i·fuku}}	& \Large{Kleider}\\
	&& \\
{\fontspec{Sazanami Gothic}稽古着}	&kei·ko·gi	& ?\\
{\fontspec{Sazanami Gothic}帯}	&obi	& Gürtel\\
{\fontspec{Sazanami Gothic}白帯}	&shiro·obi	& Weißer Gürtel\\
{\fontspec{Sazanami Gothic}黑帯}	&kuro·obi	& Schwarzer Gürtel\\
{\fontspec{Sazanami Gothic}足袋}	&tabi	& Halbstrumpf mit aprte großes Zehe\\
{\fontspec{Sazanami Gothic}草履}	&zori	& Flipflops\\
{\fontspec{Sazanami Gothic}袴}	&hakama	& traditionelle japanische Hose\\
\end{tabular}
\vspace{.5cm}
}}

\def\gkakobogaku{{
\noindent\par\begin{tabular}{@{}p{2.75cm}p{5cm}p{8.25cm}@{}}
{\fontspec{Sazanami Gothic}\Large\bfseries 解剖学}	&\Large{\bfseries{kako·b\=o·gaku}}	& \Large{Anatomie}\\
	&& \\
{\fontspec{Sazanami Gothic}腹}	&hara	& Zentrum\\
{\fontspec{Sazanami Gothic}体}	&tai	& Körper\\
{\fontspec{Sazanami Gothic}正面}	&sh\=o·men	& Stirne\\
{\fontspec{Sazanami Gothic}横面}	&yoko·men	& Schläfe\\
{\fontspec{Sazanami Gothic}膝}	&hiza	& Knie\\
{\fontspec{Sazanami Gothic}首}	&kubi	& Nacken\\
{\fontspec{Sazanami Gothic}胸}	&mune	& Brust\\
{\fontspec{Sazanami Gothic}肩}	&kata	& Schulter\\
{\fontspec{Sazanami Gothic}肘}	&hiji	& Ellbogen\\
{\fontspec{Sazanami Gothic}腕}	&ude	& Arm\\
{\fontspec{Sazanami Gothic}手首}	&te·kubi	& Handgelenk\\
{\fontspec{Sazanami Gothic}手}	&te	& Hand\\
{\fontspec{Sazanami Gothic}手刀}	&te·gatana	& Schwerthand\\
{\fontspec{Sazanami Gothic}足}	&ashi	& Bein, Fuß\\
{\fontspec{Sazanami Gothic}足首}	&asho kubi	& Knöchel\\
{\fontspec{Sazanami Gothic}腰}	&koshi	& Hüfte\\
{\fontspec{Sazanami Gothic}襟}	&eri	& Kragen\\
{\fontspec{Sazanami Gothic}身}	&mi	& Körper\\
{\fontspec{Sazanami Gothic}袖}	&sode	& ?\\
\end{tabular}
\vspace{.5cm}
}}

\def\g?{{
\noindent\par\begin{tabular}{@{}p{2.75cm}p{5cm}p{8.25cm}@{}}
{\fontspec{Sazanami Gothic}\Large\bfseries ?}	&\Large{\bfseries{?}}	& \Large{Diversen}\\
	&& \\
{\fontspec{Sazanami Gothic}合抜け}	&ai·nuke	& ?\\
{\fontspec{Sazanami Gothic}合打ち}	&ai·uchi	& ?\\
{\fontspec{Sazanami Gothic}当て身}	&a·te·mi	& ?\\
{\fontspec{Sazanami Gothic}武道}	&budo	& ?\\
{\fontspec{Sazanami Gothic}組杖}	&kumi·j\=o	& ?\\
{\fontspec{Sazanami Gothic}組太刀}	&kumi·ta·chi	& ?\\
{\fontspec{Sazanami Gothic}太刀取り}	&ta·chi·do·ri	& ?\\
{\fontspec{Sazanami Gothic}短刀取り}	&tan·to·do·ri	& ?\\
{\fontspec{Sazanami Gothic}正眼}	&sei·gan	& ?\\
{\fontspec{Sazanami Gothic}構え}	&kama·e	& ?\\
{\fontspec{Sazanami Gothic}間合い}	&ma·a·i	& ?\\
{\fontspec{Sazanami Gothic}膝行}	&shikkou	& ?\\
{\fontspec{Sazanami Gothic}太鼓}	&tai·ko	& ?\\
\end{tabular}
\vspace{.5cm}
}}

\def\g?{{
\noindent\par\begin{tabular}{@{}p{2.75cm}p{5cm}p{8.25cm}@{}}
{\fontspec{Sazanami Gothic}\Large\bfseries 方}	&\Large{\bfseries{?}}	& \Large{Richtung}\\
	&& \\
{\fontspec{Sazanami Gothic}左}	&hidari	& links\\
{\fontspec{Sazanami Gothic}右}	&migi	& rechts\\
{\fontspec{Sazanami Gothic}入身}	&iri·mi	& eintreten\\
{\fontspec{Sazanami Gothic}回転}	&kai·ten	& wegdrehen\\
{\fontspec{Sazanami Gothic}前}	&mae	& vorwärts\\
{\fontspec{Sazanami Gothic}後ろ}	&ushi·ro	& rückwärts\\
{\fontspec{Sazanami Gothic}横}	&yoko	& seitwärts\\
{\fontspec{Sazanami Gothic}内}	&uchi	& innen, inwärts\\
{\fontspec{Sazanami Gothic}外}	&soto	& außen, außerwärts\\
{\fontspec{Sazanami Gothic}斜め}	&nana·me	& ?\\
{\fontspec{Sazanami Gothic}直立}	&choku·ritsu	&  [Vetikal]\\
{\fontspec{Sazanami Gothic}水平}	&sui·hei	&  [Horizontal]\\
{\fontspec{Sazanami Gothic}立て}	&ta·te	& ?\\
{\fontspec{Sazanami Gothic}反対}	&han·tai	& gegenuber\\
{\fontspec{Sazanami Gothic}八方}	&hap·p\=o	& alle Richtungen [acht Richtungen]\\
\end{tabular}
\vspace{.5cm}
}}

\def\gundo{{
\noindent\par\begin{tabular}{@{}p{2.75cm}p{5cm}p{8.25cm}@{}}
{\fontspec{Sazanami Gothic}\Large\bfseries 運動}	&\Large{\bfseries{un·d\=o}}	& \Large{Übung}\\
	&& \\
{\fontspec{Sazanami Gothic}一教運動}	&ik·kyo undo	& ikky\=o Übung\\
{\fontspec{Sazanami Gothic}体の変更}	&tai no hen·kou	& ?\\
\end{tabular}
\vspace{.5cm}
}}

\def\ghojonokata{{
\noindent\par\begin{tabular}{@{}p{2.75cm}p{5cm}p{8.25cm}@{}}
{\fontspec{Sazanami Gothic}\Large\bfseries 法定之形}	&\Large{\bfseries{h\=o·j\=o no kata}}	& \Large{?}\\
	&& \\
{\fontspec{Sazanami Gothic}春の太刀}	&haru no ta·chi	&  [Früling Schwert]\\
{\fontspec{Sazanami Gothic}夏の太刀}	&natsu no ta·chi	&  [Sommer Schwert]\\
{\fontspec{Sazanami Gothic}秋の太刀}	&aki no ta·chi	&  [Herbst Schwert]\\
{\fontspec{Sazanami Gothic}冬の太刀}	&fuyu no ta·chi	&  [Winter Schwert]\\
{\fontspec{Sazanami Gothic}八相発破}	&hass\=o happa	& acht Directionen\\
{\fontspec{Sazanami Gothic}一刀両断}	&itto ry\=o·dan	& ?\\
{\fontspec{Sazanami Gothic}右転左転}	&u·ten sa·ten	& ?\\
{\fontspec{Sazanami Gothic}長短一身}	&cho·tan ichi·mi	& Kurz und Lang sind eins\\
{\fontspec{Sazanami Gothic}受ける }	&u·ke·ru	&  [entfangen]\\
{\fontspec{Sazanami Gothic}体剣}	&tai·ken	&  [Körper Schwert]\\
{\fontspec{Sazanami Gothic}目礼}	&moku·rei	& fracesco\\
{\fontspec{Sazanami Gothic}抜剣}	&bak·ken	&  [?]\\
{\fontspec{Sazanami Gothic}合?}	&ai seigan	& fracesco\\
{\fontspec{Sazanami Gothic}促進}	&soku·shin	&  [?]\\
{\fontspec{Sazanami Gothic}仁王太刀}	&ni·\=o da·chi	&  [?]\\
{\fontspec{Sazanami Gothic}?}	&kazashi	&  [?]\\
{\fontspec{Sazanami Gothic}合上段}	&ai j\=o·dan	&  [?]\\
{\fontspec{Sazanami Gothic}?}	&tsume	&  [?]\\
{\fontspec{Sazanami Gothic}打ち込み}	&u·chi ko·mi	& ?\\
{\fontspec{Sazanami Gothic}両腕}	&mor\=o·de	& ?\\
{\fontspec{Sazanami Gothic}押し込み}	&o·shi ko·mi	& ?\\
{\fontspec{Sazanami Gothic}一文字}	&ichi·mon·ji	& ?\\
{\fontspec{Sazanami Gothic}?}	&so tai	&  [?]\\
{\fontspec{Sazanami Gothic}体当り}	&tai ata·ri	& ?\\
{\fontspec{Sazanami Gothic}?}	&hi tachi	& ?\\
{\fontspec{Sazanami Gothic}霞}	&kasumi	& ?\\
{\fontspec{Sazanami Gothic}一重み}	&hi·toe·mi	& ?\\
{\fontspec{Sazanami Gothic}?}	&muki	&  [?]\\
{\fontspec{Sazanami Gothic}切り}	&ki·ri	&  [?]\\
{\fontspec{Sazanami Gothic}止め}	&to·me	&  [?]\\
{\fontspec{Sazanami Gothic}裏打ち}	&ura u·chi	&  [?]\\
{\fontspec{Sazanami Gothic}曙光}	&sho·k\=o	&  [?]\\
{\fontspec{Sazanami Gothic}大}	&dai	&  [groß]\\
{\fontspec{Sazanami Gothic}退けん}	&no·ke·n	&  [?]\\
\end{tabular}
\vspace{.5cm}
}}

\def\ggenkikai{{
\noindent\par\begin{tabular}{@{}p{2.75cm}p{5cm}p{8.25cm}@{}}
{\fontspec{Sazanami Gothic}\Large\bfseries 元気会}	&\Large{\bfseries{gen·ki·kai}}	& \Large{Gruppe der gesundheits Übungen}\\
	&& \\
{\fontspec{Sazanami Gothic}大円呼吸法}	&dai en ko·ky\=u h\=o	&  [?]\\
{\fontspec{Sazanami Gothic}守有の呼吸}	&su·u no ko·ky\=u	&  [?]\\
{\fontspec{Sazanami Gothic}陽の手呼吸}	&yo no te ko·ky\=u	&  [?]\\
{\fontspec{Sazanami Gothic}陰の手呼吸}	&in no te ko·ky\=u	&  [?]\\
{\fontspec{Sazanami Gothic}気結びの手呼吸}	&ki·musu·bi no te ko·ky\=u	&  [?]\\
{\fontspec{Sazanami Gothic}阿吽の呼吸}	&a·un no ko·ky\=u	&  [?]\\
{\fontspec{Sazanami Gothic}元の呼吸}	&gen no ko·ky\=u	&  [?]\\
{\fontspec{Sazanami Gothic}寝運動}	&ne un·d\=o	&  [?]\\
{\fontspec{Sazanami Gothic}揺動法}	&y\=o d\=o·h\=o	&  [?]\\
{\fontspec{Sazanami Gothic}毛管運動}	&m\=o·kan un·d\=o	&  [?]\\
{\fontspec{Sazanami Gothic}合掌合蹠運動}	&gas·sh\=o gas·seki un·d\=o	&  [?]\\
{\fontspec{Sazanami Gothic}金魚運動}	&kin·gyo un·d\=o	&  [?]\\
{\fontspec{Sazanami Gothic}馬運動}	&uma un·d\=o	&  [?]\\
\end{tabular}
\vspace{.5cm}
}}

