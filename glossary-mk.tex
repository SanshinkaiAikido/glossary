\def\gryu{{
\noindent\par\begin{tabular}{@{}p{2.75cm}p{5cm}p{8.25cm}@{}}
{\fontspec{Sazanami Gothic}\Large\bfseries 竜}	&\Large{\bfseries{ry\=u}}	& \Large{тип, начин}\\
	&& \\
{\fontspec{Sazanami Gothic}合気道}	&ai·ki·d\=o	& начин да ја хармонизираш ки енергијата\\
{\fontspec{Sazanami Gothic}合気?}	&ai·ki·jutsu	& ?\\
{\fontspec{Sazanami Gothic}合気杖}	&ai·ki·j\=o	& техники со стап каде што се употрбува аики принципот\\
{\fontspec{Sazanami Gothic}合気剣}	&ai·ki·ken	& техники со меч каде што се употрбува аики принципот\\
{\fontspec{Sazanami Gothic}法定形}	&ho·jo kata	& стил на мечување\\
{\fontspec{Sazanami Gothic}鹿島神傳直心影流}	&Kashima Shinden Jikishinkage-ry\=u	& стил на мечување\\
{\fontspec{Sazanami Gothic}元気会}	&gen·ki·kai	& ?\\
{\fontspec{Sazanami Gothic}整体}	&sei·tai	& ?\\
\end{tabular}
\vspace{.5cm}
}}

\def\gkeikoho{{
\noindent\par\begin{tabular}{@{}p{2.75cm}p{5cm}p{8.25cm}@{}}
{\fontspec{Sazanami Gothic}\Large\bfseries ?}	&\Large{\bfseries{keiko h\=o}}	& \Large{TODOдух}\\
	&& \\
{\fontspec{Sazanami Gothic}?稽古}	&kakari kei·ko	& наизменично вежбање со два напаѓачи без пауза?\\
{\fontspec{Sazanami Gothic}?}	&kigata	& меко изведување на техники\\
{\fontspec{Sazanami Gothic}鍛錬}	&tan·ren	& тврдо, стегнато изведување на техники\\
{\fontspec{Sazanami Gothic}真剣}	&shin·ken	& реалистично изведување на техники\\
{\fontspec{Sazanami Gothic}崩し}	&kuzushi	& ?\\
{\fontspec{Sazanami Gothic}後ろ技}	&ushi·ro·waza	& ?\\
{\fontspec{Sazanami Gothic}自由技}	&ji·y\=u waza	& ?\\
{\fontspec{Sazanami Gothic}乱取り}	&ran·do·ri	& ?\\
{\fontspec{Sazanami Gothic}素振り}	&su·bu·ri	&  [?]\\
\end{tabular}
\vspace{.5cm}
}}

\def\g?{{
\noindent\par\begin{tabular}{@{}p{2.75cm}p{5cm}p{8.25cm}@{}}
{\fontspec{Sazanami Gothic}\Large\bfseries ?}	&\Large{\bfseries{?}}	& \Large{позиција}\\
	&& \\
{\fontspec{Sazanami Gothic}座り技}	&suwa·ri waza	& техники кои се изведуваат движејќи се на колена\\
{\fontspec{Sazanami Gothic}半身半立ち技}	&han·mi han·da·chi waza	& техники при кои тори-то седи а уке-то стои\\
{\fontspec{Sazanami Gothic}立ち技}	&ta·chi waza	& техники кои се изведуваат во стоечки став\\
{\fontspec{Sazanami Gothic}座}	&za	& седење\\
{\fontspec{Sazanami Gothic}正座}	&sei·za	& седење на колена, подколениците и стапалата со предната страна залепени за подот\\
{\fontspec{Sazanami Gothic}跪座}	&ki·za	& седење на колена, стапалата подигнати на прсти\\
{\fontspec{Sazanami Gothic}割座}	&wari·za	& седење на колена а колковите на подот меѓу стопалата\\
{\fontspec{Sazanami Gothic}?座}	&cho·za	& седење со нозете испружени напред\\
{\fontspec{Sazanami Gothic}?座}	&kai·za	& седење со испружени и раширени нозе\\
{\fontspec{Sazanami Gothic}体座}	&tai·za	& седење со колената допрени на гради а стапалата на под\\
{\fontspec{Sazanami Gothic}?座}	&gaku·za	& седење со споени стопала а колената странично на под\\
{\fontspec{Sazanami Gothic}安座}	&an·za	& турско седење\\
{\fontspec{Sazanami Gothic}八相(八双)}	&hass\=o	& ?\\
{\fontspec{Sazanami Gothic}?座}	&kahuza	& јога седење, позиција лотус\\
\end{tabular}
\vspace{.5cm}
}}

\def\gaite{{
\noindent\par\begin{tabular}{@{}p{2.75cm}p{5cm}p{8.25cm}@{}}
{\fontspec{Sazanami Gothic}\Large\bfseries 相手}	&\Large{\bfseries{ai·te}}	& \Large{партнер}\\
	&& \\
{\fontspec{Sazanami Gothic}多人数取り}	&ta·nin·z\=u to·ri	& тројца или повеќе напаѓаат истовремено, а се изведува само почетокот од техниката\\
{\fontspec{Sazanami Gothic}掛り稽古}	&ka·ka·ri gei·ko	& наизменично вежбање со два напаѓачи без пауза\\
\end{tabular}
\vspace{.5cm}
}}

\def\gbuki{{
\noindent\par\begin{tabular}{@{}p{2.75cm}p{5cm}p{8.25cm}@{}}
{\fontspec{Sazanami Gothic}\Large\bfseries 武器}	&\Large{\bfseries{bu·ki}}	& \Large{оружје}\\
	&& \\
{\fontspec{Sazanami Gothic}杖}	&j\=o	& дрвен стап\\
{\fontspec{Sazanami Gothic}棒}	&b\=o	& долг дрвен стап\\
{\fontspec{Sazanami Gothic}木剣}	&bok·ken, boku·t\=o	& закривен дрвен меч\\
{\fontspec{Sazanami Gothic}短刀}	&tan·t\=o	& нож\\
{\fontspec{Sazanami Gothic}脇差}	&waki·zashi, shoto	& краток самурајски закривен меч\\
{\fontspec{Sazanami Gothic}刀}	&(ken?,) katana	& долг самурајски закривен меч\\
{\fontspec{Sazanami Gothic}太刀}	&ta·chi	& долг меч\\
{\fontspec{Sazanami Gothic}剣}	&ken	& меч\\
{\fontspec{Sazanami Gothic}日本刀}	&ni·hon·t\=o	& Јапонски меч\\
{\fontspec{Sazanami Gothic}薙刀}	&naginata	& хелебарда, вид на копје со широко сечиво на врвот\\
{\fontspec{Sazanami Gothic}槍}	&yari	& копје\\
{\fontspec{Sazanami Gothic}鍔}	&tsuba	& штитник пред дршката на мечот\\
{\fontspec{Sazanami Gothic}鞘}	&saya	& ?\\
{\fontspec{Sazanami Gothic}?}	&bugukake	& држач за оружје\\
\end{tabular}
\vspace{.5cm}
}}

\def\gkokukegi{{
\noindent\par\begin{tabular}{@{}p{2.75cm}p{5cm}p{8.25cm}@{}}
{\fontspec{Sazanami Gothic}\Large\bfseries 迫撃}	&\Large{\bfseries{koku·kegi}}	& \Large{напад}\\
	&& \\
{\fontspec{Sazanami Gothic}気合せ正面打ち}	&ki·a·wa·se sh\=o·men u·chi	& удар во глава со отворена дланка поаѓајќи од колковите\\
{\fontspec{Sazanami Gothic}片手取り相半身}	&kata·te do·ri ai han·mi	& фат за зглобот над дланката на соодветно истата рака, десна за десна и лева за лева\\
{\fontspec{Sazanami Gothic}片手取り逆半身}	&kata·te do·ri gyaku han·mi	& фат за зглобот над дланката на соодветно спротивната рака, десна за лева и лева за десна\\
{\fontspec{Sazanami Gothic}両手取り}	&ry\=o·te do·ri	& фат со две раце за двете раце\\
{\fontspec{Sazanami Gothic}正面打ち}	&sh\=o·men u·chi	& удар во предниот дел на главата од горе\\
{\fontspec{Sazanami Gothic}片手両手取り}	&kata·te ry\=o·te do·ri	& фат со двете раце за една рака\\
{\fontspec{Sazanami Gothic}肩取り}	&kata do·ri	& фат за рамо\\
{\fontspec{Sazanami Gothic}後ろ両手取り}	&ushi·ro ry\=o·te do·ri	& фат со две раце за двете раце од зад грб\\
{\fontspec{Sazanami Gothic}横面打ち}	&yoko·men u·chi	& удар во главата од страна\\
{\fontspec{Sazanami Gothic}中段突き}	&ch\=u·dan tsu·ki	& директен удар во средишниот дел на телото\\
{\fontspec{Sazanami Gothic}肩取り面打ち}	&kata do·ri men u·chi	& фат за рамо со едната рака и удар со другата во глава\\
{\fontspec{Sazanami Gothic}後ろ両肩取り}	&ushi·ro ry\=o·kata do·ri	& фат со две раце за двете рамења од зад грб\\
{\fontspec{Sazanami Gothic}上段突き}	&j\=o·dan tsu·ki	& директен удар во горниот дел на телото\\
{\fontspec{Sazanami Gothic}両襟取り}	&ry\=o·eri do·ri	& фат со две раце за крагната\\
{\fontspec{Sazanami Gothic}蹴り}	&ke·ri	& удар со нога\\
{\fontspec{Sazanami Gothic}後ろ首絞め}	&ushi·ro kubi jime	& давење од зад грб\\
\end{tabular}
\vspace{.5cm}
}}

\def\guchi/soto{{
\noindent\par\begin{tabular}{@{}p{2.75cm}p{5cm}p{8.25cm}@{}}
{\fontspec{Sazanami Gothic}\Large\bfseries 内/外}	&\Large{\bfseries{uchi/soto}}	& \Large{одвнатре/однадвор}\\
	&& \\
{\fontspec{Sazanami Gothic}内捌き}	&uchi saba·ki	& движење кон внатре\\
{\fontspec{Sazanami Gothic}外捌き}	&soto saba·ki	& движење кон надвор\\
\end{tabular}
\vspace{.5cm}
}}

\def\gtaisabaki{{
\noindent\par\begin{tabular}{@{}p{2.75cm}p{5cm}p{8.25cm}@{}}
{\fontspec{Sazanami Gothic}\Large\bfseries 体捌き}	&\Large{\bfseries{tai saba·ki}}	& \Large{движење на телото}\\
	&& \\
{\fontspec{Sazanami Gothic}前足入身}	&mae ashi iri·mi	& директен влез право напред со предната нога\\
{\fontspec{Sazanami Gothic}後ろ足入身}	&ushi·ro ashi iri·mi	& директен влез право напред со задната нога\\
{\fontspec{Sazanami Gothic}前足転換}	&mae ashi ten·kan	& избегнувачко движење, предната нога прави полукруг\\
{\fontspec{Sazanami Gothic}後ろ足転換}	&ushi·ro ashi ten·kan	& избегнувачко движење со свртување на предната нога, задната нога прави полукруг\\
{\fontspec{Sazanami Gothic}前足?}	&mae ashi tenshin	& избегнувачко движење со повлекување на предната нога\\
{\fontspec{Sazanami Gothic}後ろ足?}	&ushi·ro ashi tenshin	& избегнувачко движење со повлекување на задната нога\\
{\fontspec{Sazanami Gothic}前足入身転換}	&mae ashi iri·mi ten·kan	& комбинирано движење за избегнување, започнато со предната нога\\
{\fontspec{Sazanami Gothic}後ろ足入身転換}	&ushi·ro ashi iri·mi ten·kan	& комбинирано движење за избегнување, започнато со задната нога\\
{\fontspec{Sazanami Gothic}前足入身転換?}	&mae ashi iri·mi ten·kan tenshin	& комбинирано движење за двојно избегнување, започнато со предната нога и завршува со повлекување на задната нога\\
{\fontspec{Sazanami Gothic}後ろ足入身転換?}	&ushi·ro ashi iri·mi ten·kan tenshin	& комбинирано движење за двојно избегнување, започнато со задната нога и завршува со повлекување на задната нога\\
{\fontspec{Sazanami Gothic}前足転換?}	&mae ashi ten·kan tenshin	& комбинирано движење\\
{\fontspec{Sazanami Gothic}後ろ足転換?}	&ushi·ro ashi ten·kan tenshin	& комбинирано движење\\
{\fontspec{Sazanami Gothic}後ろ足入身?}	&ushi·ro ashi iri·mi tenkai	& ?\\
\end{tabular}
\vspace{.5cm}
}}

\def\gtesabaki{{
\noindent\par\begin{tabular}{@{}p{2.75cm}p{5cm}p{8.25cm}@{}}
{\fontspec{Sazanami Gothic}\Large\bfseries 手捌き}	&\Large{\bfseries{te saba·ki}}	& \Large{?}\\
	&& \\
{\fontspec{Sazanami Gothic}上半円}	&kami han·en	& горен вертикален полукруг\\
{\fontspec{Sazanami Gothic}下半円}	&shimo han·en	& долен вертикален полукруг\\
{\fontspec{Sazanami Gothic}手首返し}	&te·kubi gae·shi	& ?\\
{\fontspec{Sazanami Gothic}十字結び}	&jy\=u·ji musu·bi	& ?\\
{\fontspec{Sazanami Gothic}受け流し}	&uke naga·shi	& ?\\
\end{tabular}
\vspace{.5cm}
}}

\def\gnote{{
\noindent\par\begin{tabular}{@{}p{2.75cm}p{5cm}p{8.25cm}@{}}
{\fontspec{Sazanami Gothic}\Large\bfseries の手}	&\Large{\bfseries{no·te}}	& \Large{рака}\\
	&& \\
{\fontspec{Sazanami Gothic}打ちの手}	&u·chi·no·te	& рака што удира\\
{\fontspec{Sazanami Gothic}肩の手}	&kata·no·te	& рака што фаќа за рамо\\
\end{tabular}
\vspace{.5cm}
}}

\def\gdan{{
\noindent\par\begin{tabular}{@{}p{2.75cm}p{5cm}p{8.25cm}@{}}
{\fontspec{Sazanami Gothic}\Large\bfseries 段}	&\Large{\bfseries{dan}}	& \Large{ниво, височина}\\
	&& \\
{\fontspec{Sazanami Gothic}下段}	&ge·dan	& долно\\
{\fontspec{Sazanami Gothic}中段}	&ch\=u·dan	& средишно\\
{\fontspec{Sazanami Gothic}上段}	&j\=o·dan	& горно\\
{\fontspec{Sazanami Gothic}脇構}	&waki gamae	& долно\\
{\fontspec{Sazanami Gothic}下段の構え}	&ge·dan no kama·e	& долно\\
{\fontspec{Sazanami Gothic}中段の構え}	&ch\=u·dan no kama·e	& средишно\\
{\fontspec{Sazanami Gothic}上段の構}	&j\=o·dan no kamae	& горно\\
\end{tabular}
\vspace{.5cm}
}}

\def\gwaza{{
\noindent\par\begin{tabular}{@{}p{2.75cm}p{5cm}p{8.25cm}@{}}
{\fontspec{Sazanami Gothic}\Large\bfseries 技}	&\Large{\bfseries{waza}}	& \Large{техника}\\
	&& \\
{\fontspec{Sazanami Gothic}一教}	&ik·ky\=o	&  [прво учење, прв принцип]\\
{\fontspec{Sazanami Gothic}二教}	&ni·ky\=o	&  [второ учење, втор принцип]\\
{\fontspec{Sazanami Gothic}三教}	&san·ky\=o	&  [трето учење, трет принцип]\\
{\fontspec{Sazanami Gothic}四教}	&yon·ky\=o	&  [четврто учење, четврт принцип]\\
{\fontspec{Sazanami Gothic}隅落し}	&sumi·oto·shi	& спуштање во агол, техника на фрлање\\
{\fontspec{Sazanami Gothic}小手返し}	&ko·te·gae·shi	& фрлање со завртување на раката во спротивна насока\\
{\fontspec{Sazanami Gothic}入身投げ}	&iri·mi·na·ge	& фрлање со директен влез\\
{\fontspec{Sazanami Gothic}四方投げ}	&shi·h\=o·na·ge	& фрлање во четири правци\\
{\fontspec{Sazanami Gothic}五教}	&go·ky\=o	& петто учење, петти принцип\\
{\fontspec{Sazanami Gothic}肘決め抑え}	&hiji ki·me osa·e	& фиксација со лост на лактот\\
{\fontspec{Sazanami Gothic}内回転三教}	&uchi·kai·ten·san·ky\=o	& третиот принцип со поминување под рака\\
{\fontspec{Sazanami Gothic}腕絡み}	&ude gara·mi	& начин на лост на раката\\
{\fontspec{Sazanami Gothic}合気腰}	&ai·ki·goshi	& фрлање преку колкови со искористување на инерцијата на уке-то\\
{\fontspec{Sazanami Gothic}合気落し}	&ai·ki·oto·shi	&  [спуштање со потфаќање и подигнување на нозете]\\
{\fontspec{Sazanami Gothic}回転投げ}	&kai·ten·na·ge	&  [кружно, ротирачко фрлање]\\
{\fontspec{Sazanami Gothic}腕決め投げ}	&ude ki·me na·ge	&  [фрлање корисејќи лост на лактот]\\
{\fontspec{Sazanami Gothic}前落し}	&mae·oto·shi	&  [напредно спуштање / горе-долу]\\
{\fontspec{Sazanami Gothic}引き落し}	&hiki·oto·shi	&  [напредно спуштање / лево-десно]\\
{\fontspec{Sazanami Gothic}切り落し}	&ki·ri oto·shi	&  [напредно спуштање / напред-назад]\\
{\fontspec{Sazanami Gothic}回転落し}	&kai·ten·oto·shi	& напредно спуштање / спирално\\
{\fontspec{Sazanami Gothic}天地投げ}	&ten chi na·ge	& небо-земја фрлање\\
{\fontspec{Sazanami Gothic}玄形呼吸投げ}	&gen·kei·ko·ky\=u·na·ge	& фрлање со издишување спротивно на небо-земја фрлањето\\
{\fontspec{Sazanami Gothic}内回転投げ}	&uchi·kai·ten·na·ge	& кружно, ротирачко фрлање со поминување под рака\\
{\fontspec{Sazanami Gothic}振り突き呼吸投げ}	&furi·zu·ki ko·ky\=u na·ge	& фрлање произлезено од директен влез со рака во лице\\
{\fontspec{Sazanami Gothic}十字絡み}	&jy\=u·ji gara·mi	&  [фрлање со вкрстување на рацете]\\
{\fontspec{Sazanami Gothic}鳥り船呼吸投げ}	&to·ri fune ko·ky\=u·na·ge	&  [икјо фрлање]\\
{\fontspec{Sazanami Gothic}四方蹴り呼吸投げ}	&shi·h\=o ge·ri ko·ky\=u·na·ge	&  [?]\\
{\fontspec{Sazanami Gothic}教絡み三教投げ}	&ude gara·mi san·ky\=o na·ge	& фрлање со лостот удегарами во санкјо варијација\\
{\fontspec{Sazanami Gothic}教絡み四教投げ}	&ude gara·mi yon·ky\=o na·ge	& фрлање со лостот удегарами во јонкјо варијација\\
{\fontspec{Sazanami Gothic}腰投げ}	&koshi na·ge	& фрлање преку колкови\\
{\fontspec{Sazanami Gothic}外回転投げ}	&soto kai·ten·na·ge	& надворешмо кружно фрлање\\
{\fontspec{Sazanami Gothic}斬刀呼吸投げ}	&zan·to ko·ky\=u na·ge	& занто фрлање со издишна сила\\
{\fontspec{Sazanami Gothic}教絡み抑え}	&ude gara·mi osa·e	& уде гарами фиксација\\
{\fontspec{Sazanami Gothic}手車}	&te guruma	& форма на завртено фрлање со рака\\
{\fontspec{Sazanami Gothic}背負い車}	&se·o·i guruma	& форма на завртено фрлање преку рамо\\
{\fontspec{Sazanami Gothic}腰車}	&koshi guruma	& форма на завртено фрлање преку колкови\\
{\fontspec{Sazanami Gothic}沈身腰車}	&chin shin koshi guruma	& ?\\
\end{tabular}
\vspace{.5cm}
}}

\def\gomote/ura{{
\noindent\par\begin{tabular}{@{}p{2.75cm}p{5cm}p{8.25cm}@{}}
{\fontspec{Sazanami Gothic}\Large\bfseries 表/裏}	&\Large{\bfseries{omote/ura}}	& \Large{од лице/од опачина}\\
	&& \\
{\fontspec{Sazanami Gothic}表}	&omote	& од лице (предна страна)\\
{\fontspec{Sazanami Gothic}裏}	&ura	& од опачина (задна страна)\\
\end{tabular}
\vspace{.5cm}
}}

\def\gyin/yang{{
\noindent\par\begin{tabular}{@{}p{2.75cm}p{5cm}p{8.25cm}@{}}
{\fontspec{Sazanami Gothic}\Large\bfseries ?/?}	&\Large{\bfseries{yin/yang}}	& \Large{јин/јанг}\\
	&& \\
{\fontspec{Sazanami Gothic}?}	&yin	& јин\\
{\fontspec{Sazanami Gothic}?}	&yang	& јанг\\
\end{tabular}
\vspace{.5cm}
}}

\def\gri{{
\noindent\par\begin{tabular}{@{}p{2.75cm}p{5cm}p{8.25cm}@{}}
{\fontspec{Sazanami Gothic}\Large\bfseries 理}	&\Large{\bfseries{ri}}	& \Large{принцип}\\
	&& \\
{\fontspec{Sazanami Gothic}水}	&su, mizu	& вода, горе-доле, исток\\
{\fontspec{Sazanami Gothic}土}	&do, tsu	& земја, лево-десно, југ\\
{\fontspec{Sazanami Gothic}風}	&hu	& ветер, напред-назад, запад\\
{\fontspec{Sazanami Gothic}火}	&ka, hi	& оган, спирала, север\\
{\fontspec{Sazanami Gothic}?}	&complete?	& човек\\
{\fontspec{Sazanami Gothic}春}	&haru	& пролет\\
{\fontspec{Sazanami Gothic}夏}	&natsu	& лето\\
{\fontspec{Sazanami Gothic}秋}	&aki	& есен\\
{\fontspec{Sazanami Gothic}冬}	&fuyu	& зима\\
{\fontspec{Sazanami Gothic}攻防の原理}	&k\=o·b\=o no gen·ri	& ?\\
{\fontspec{Sazanami Gothic}打ちの理}	&u·chi no ri	& ?\\
{\fontspec{Sazanami Gothic}抑えの理}	&osa·e no ri	& ?\\
{\fontspec{Sazanami Gothic}投げの理}	&na·ge no ri	& ?\\
{\fontspec{Sazanami Gothic}斬の理}	&zan no ri	& ?\\
\end{tabular}
\vspace{.5cm}
}}

\def\gnage/osae{{
\noindent\par\begin{tabular}{@{}p{2.75cm}p{5cm}p{8.25cm}@{}}
{\fontspec{Sazanami Gothic}\Large\bfseries 投げ/抑え}	&\Large{\bfseries{nage/osae}}	& \Large{фрлање/фиксација}\\
	&& \\
{\fontspec{Sazanami Gothic}投げ}	&na·ge	& фрлање\\
{\fontspec{Sazanami Gothic}抑え}	&osa·e	& фиксација\\
{\fontspec{Sazanami Gothic}投げ抑え}	&na·ge osa·e	& фиксација фиксација\\
\end{tabular}
\vspace{.5cm}
}}

\def\gukemi{{
\noindent\par\begin{tabular}{@{}p{2.75cm}p{5cm}p{8.25cm}@{}}
{\fontspec{Sazanami Gothic}\Large\bfseries 受身}	&\Large{\bfseries{u·ke·mi}}	& \Large{примање со телото, пад}\\
	&& \\
	&boven onder u·ke·mi	& ?\\
{\fontspec{Sazanami Gothic}前受身}	&mae u·ke·mi	& пад напред\\
{\fontspec{Sazanami Gothic}後ろ受身}	&ushi·ro u·ke·mi	& пад назад\\
{\fontspec{Sazanami Gothic}横受身}	&yoko u·ke·mi	& пад на страна\\
	&chokuto	& ?\\
{\fontspec{Sazanami Gothic}飛び受身}	&to·bi u·ke·mi	& слободен пад\\
\end{tabular}
\vspace{.5cm}
}}

\def\g?{{
\noindent\par\begin{tabular}{@{}p{2.75cm}p{5cm}p{8.25cm}@{}}
{\fontspec{Sazanami Gothic}\Large\bfseries ?}	&\Large{\bfseries{?}}	& \Large{личност}\\
	&& \\
{\fontspec{Sazanami Gothic}植芝 盛平}	&Ueshiba, Morihei	& основачот на аикидото, \=osensei (14-12-1883 - 26-04-1969)\\
{\fontspec{Sazanami Gothic}植芝 吉祥丸}	&Ueshiba, Kisshomaru	& син на основачот, втор дошу (27-06-1921 - 04-01-1999)\\
{\fontspec{Sazanami Gothic}植芝 守央}	&Moriteru Ueshiba	& внук на основачот, трет дошу (02-04-1951)\\
{\fontspec{Sazanami Gothic}多田 宏}	&Tada, Hiroshi	& ученик на \=osensei (13-12-1929)\\
{\fontspec{Sazanami Gothic}池田 昌富}	&Ikeda, Masatomi	& ученик на Тада шиханот, shihan (08-04-1940)\\
\end{tabular}
\vspace{.5cm}
}}

\def\gkazueru{{
\noindent\par\begin{tabular}{@{}p{2.75cm}p{5cm}p{8.25cm}@{}}
{\fontspec{Sazanami Gothic}\Large\bfseries 数える}	&\Large{\bfseries{kazu·e·ru}}	& \Large{броење}\\
	&& \\
{\fontspec{Sazanami Gothic}○}	&zero	& нула\\
{\fontspec{Sazanami Gothic}一}	&ichi	& еден\\
{\fontspec{Sazanami Gothic}二}	&ni	& два\\
{\fontspec{Sazanami Gothic}三}	&san	& три\\
{\fontspec{Sazanami Gothic}四}	&shi, yon	& четири\\
{\fontspec{Sazanami Gothic}五}	&go	& пет\\
{\fontspec{Sazanami Gothic}六}	&roku	& шест\\
{\fontspec{Sazanami Gothic}七}	&shichi, nana	& седум\\
{\fontspec{Sazanami Gothic}八}	&hachi	& осум\\
{\fontspec{Sazanami Gothic}九}	&ku, ky\=u	& девет\\
{\fontspec{Sazanami Gothic}十}	&j\=u	& десет\\
{\fontspec{Sazanami Gothic}十一}	&j\=u ichi	& единаесет\\
{\fontspec{Sazanami Gothic}二十}	&ni j\=u	& дваесет\\
{\fontspec{Sazanami Gothic}二十一}	&ni j\=u ichi	& дваесет и еден\\
{\fontspec{Sazanami Gothic}百}	&hyaku	& сто\\
{\fontspec{Sazanami Gothic}千}	&sen	& илјада\\
{\fontspec{Sazanami Gothic}万}	&man	& десет илјади\\
{\fontspec{Sazanami Gothic}一本目}	&ip·pon·me	& ?\\
{\fontspec{Sazanami Gothic}ニ本目}	&ni·hon·me	& ?\\
{\fontspec{Sazanami Gothic}三本目}	&san·bon·me	& ?\\
{\fontspec{Sazanami Gothic}四本目}	&yon·hon·me	& ?\\
{\fontspec{Sazanami Gothic}五本目}	&go·hon·me	& ?\\
{\fontspec{Sazanami Gothic}六本目}	&roku·hon·me	& ?\\
{\fontspec{Sazanami Gothic}七本目}	&nana·hon·me	& ?\\
{\fontspec{Sazanami Gothic}八本目}	&hachi·hon·me	& ?\\
\end{tabular}
\vspace{.5cm}
}}

\def\gdankai{{
\noindent\par\begin{tabular}{@{}p{2.75cm}p{5cm}p{8.25cm}@{}}
{\fontspec{Sazanami Gothic}\Large\bfseries 段階}	&\Large{\bfseries{dan·kai}}	& \Large{степени}\\
	&& \\
{\fontspec{Sazanami Gothic}段}	&dan	& мајсторски степен - дан\\
{\fontspec{Sazanami Gothic}級}	&ky\=u	& ученички степен - кју\\
{\fontspec{Sazanami Gothic}無級}	&mu·ky\=u	& без степен\\
{\fontspec{Sazanami Gothic}六級}	&rok·ky\=u	& шести кју\\
{\fontspec{Sazanami Gothic}五級}	&go·ky\=u	& петти кју\\
{\fontspec{Sazanami Gothic}四級}	&yon·ky\=u	& четврти кју\\
{\fontspec{Sazanami Gothic}参級}	&san·ky\=u	& трети кју\\
{\fontspec{Sazanami Gothic}弐級}	&ni·ky\=u	& втори кју\\
{\fontspec{Sazanami Gothic}壱級}	&ik·ky\=u	& прв кју\\
{\fontspec{Sazanami Gothic}有段者}	&y\=u·dan·sha	& носител на дан степен\\
{\fontspec{Sazanami Gothic}無段者}	&mu·dan·sha	& без дан\\
{\fontspec{Sazanami Gothic}初段}	&sho·dan	& прв дан\\
{\fontspec{Sazanami Gothic}弐段}	&ni·dan	& втор дан\\
{\fontspec{Sazanami Gothic}参段}	&san·dan	& трет дан\\
{\fontspec{Sazanami Gothic}四段}	&yon·dan	& четврт дан\\
{\fontspec{Sazanami Gothic}五段}	&go·dan	& петти дан\\
{\fontspec{Sazanami Gothic}六段}	&roku·dan	& шести дан\\
{\fontspec{Sazanami Gothic}七段}	&nana·dan	& седми дан\\
{\fontspec{Sazanami Gothic}八段}	&hachi·dan	& осми дан\\
{\fontspec{Sazanami Gothic}九段}	&ku·dan	& девети дан\\
{\fontspec{Sazanami Gothic}十段}	&jy\=u·dan	& десети дан\\
\end{tabular}
\vspace{.5cm}
}}

\def\g?{{
\noindent\par\begin{tabular}{@{}p{2.75cm}p{5cm}p{8.25cm}@{}}
{\fontspec{Sazanami Gothic}\Large\bfseries ?}	&\Large{\bfseries{?}}	& \Large{општо}\\
	&& \\
{\fontspec{Sazanami Gothic}合}	&ai	& хармонија, здружување\\
{\fontspec{Sazanami Gothic}気}	&ki	& енергија, сила\\
{\fontspec{Sazanami Gothic}道}	&d\=o	& пат, наука\\
{\fontspec{Sazanami Gothic}合気道}	&ai·ki·d\=o	& наука за силоускладување\\
{\fontspec{Sazanami Gothic}合気道家}	&ai·ki·d\=o·ka	& вежбач на аикидо\\
{\fontspec{Sazanami Gothic}合気会}	&ai·ki·kai	& Организација за промоција на аикидото(институт за силоускладување)\\
{\fontspec{Sazanami Gothic}受け}	&u·ke	& напаѓач [примач]\\
{\fontspec{Sazanami Gothic}取り}	&to·ri	& ? [?]\\
{\fontspec{Sazanami Gothic}投げ}	&na·ge	&  [?]\\
{\fontspec{Sazanami Gothic}打太刀}	&uchi·da·chi	& ?\\
{\fontspec{Sazanami Gothic}受太刀}	&shi·da·chi	& ?\\
\end{tabular}
\vspace{.5cm}
}}

\def\g?{{
\noindent\par\begin{tabular}{@{}p{2.75cm}p{5cm}p{8.25cm}@{}}
{\fontspec{Sazanami Gothic}\Large\bfseries ?}	&\Large{\bfseries{?}}	& \Large{упатства}\\
	&& \\
{\fontspec{Sazanami Gothic}お願いします}	&o·nega·i shi·ma·su	& ве молам повелете\\
{\fontspec{Sazanami Gothic}どうも有難う御座いました}	&do·o·mo a·ri·ga·to·o go·za·i·ma·shi·ta	& благодарам (на крај на вежбањето)\\
{\fontspec{Sazanami Gothic}はい}	&hai	& да\\
{\fontspec{Sazanami Gothic}止め}	&yame	& сопри\\
{\fontspec{Sazanami Gothic}始め}	&haji·me	& почни\\
{\fontspec{Sazanami Gothic}待って}	&ma·t·te	& шести кју\\
{\fontspec{Sazanami Gothic}立て}	&ta·te	& стани\\
{\fontspec{Sazanami Gothic}?}	&suwatte	& седни\\
{\fontspec{Sazanami Gothic}?}	&mawatte	& сврти се\\
{\fontspec{Sazanami Gothic}個体}	&ko·tai	& смени партнер\\
{\fontspec{Sazanami Gothic}?理解}	&mo·ri·kai	& повтори ја вежбата\\
{\fontspec{Sazanami Gothic}反対}	&han·tai	& од другата страна\\
{\fontspec{Sazanami Gothic}相手?}	&ai·te kaite	& најди друг партнер\\
{\fontspec{Sazanami Gothic}礼}	&rei	& почит, поклони се\\
{\fontspec{Sazanami Gothic}?礼}	&ritsu·rei	& стоечко поклонување\\
{\fontspec{Sazanami Gothic}座礼}	&za·rei	& седечко поклонување\\
\end{tabular}
\vspace{.5cm}
}}

\def\g?{{
\noindent\par\begin{tabular}{@{}p{2.75cm}p{5cm}p{8.25cm}@{}}
{\fontspec{Sazanami Gothic}\Large\bfseries ?}	&\Large{\bfseries{?}}	& \Large{наслови, титули}\\
	&& \\
{\fontspec{Sazanami Gothic}大先生}	&\=o·sen·sei	& големиот учител\\
{\fontspec{Sazanami Gothic}先生}	&sen·sei	& учител, инструктор\\
{\fontspec{Sazanami Gothic}道主}	&d\=o·shu	& оној кој го покажува патот, првиот човек на дисциплината\\
{\fontspec{Sazanami Gothic}師範}	&shi·han	& главен учител (со најмалку 6-ти дан)\\
{\fontspec{Sazanami Gothic}指導員}	&shi·do·in	& пошмик главен учител (4-ти или 5-ти дан)\\
{\fontspec{Sazanami Gothic}副指導員}	&fuku·shi·do·in	& пошник на шидоинот (2-ор или 3-ти дан)\\
{\fontspec{Sazanami Gothic}先輩}	&sen·pai	& старешина\\
{\fontspec{Sazanami Gothic}後輩}	&k\=o·hai	& јуниор\\
{\fontspec{Sazanami Gothic}無段者}	&mu·dan·sha	& без дан\\
{\fontspec{Sazanami Gothic}有段者}	&y\=u·dan·sha	& со дан\\
{\fontspec{Sazanami Gothic}道場長}	&d\=o·j\=o·cho	& глава на доџото\\
{\fontspec{Sazanami Gothic}弟子}	&de·shi	& ученик\\
{\fontspec{Sazanami Gothic}内弟子}	&uchi·de·shi	& ученик кој живее во доџото\\
{\fontspec{Sazanami Gothic}外弟子}	&soto·de·shi	& ученик кој не живее во доџото\\
\end{tabular}
\vspace{.5cm}
}}

\def\g?{{
\noindent\par\begin{tabular}{@{}p{2.75cm}p{5cm}p{8.25cm}@{}}
{\fontspec{Sazanami Gothic}\Large\bfseries ?}	&\Large{\bfseries{?}}	& \Large{објекти}\\
	&& \\
{\fontspec{Sazanami Gothic}道場}	&d\=o·j\=o	& салата за вежбање (место за одредена дисциплина)\\
{\fontspec{Sazanami Gothic}上座}	&kamiza	& челото на салата каде што стои сликата на основачот\\
{\fontspec{Sazanami Gothic}?座}	&shomiza	&  [?]\\
{\fontspec{Sazanami Gothic}?}	&joseki	& ?\\
{\fontspec{Sazanami Gothic}?}	&shimoseki	& ?\\
{\fontspec{Sazanami Gothic}畳}	&tatami	& душеци\\
{\fontspec{Sazanami Gothic}本部}	&hom·bu	& штаб\\
\end{tabular}
\vspace{.5cm}
}}

\def\gifuku{{
\noindent\par\begin{tabular}{@{}p{2.75cm}p{5cm}p{8.25cm}@{}}
{\fontspec{Sazanami Gothic}\Large\bfseries 衣服}	&\Large{\bfseries{i·fuku}}	& \Large{облека}\\
	&& \\
{\fontspec{Sazanami Gothic}稽古着}	&kei·ko·gi	& облека за вежбање\\
{\fontspec{Sazanami Gothic}帯}	&obi	& поајс\\
{\fontspec{Sazanami Gothic}白帯}	&shiro·obi	& бел појас\\
{\fontspec{Sazanami Gothic}黑帯}	&kuro·obi	& црн појас\\
{\fontspec{Sazanami Gothic}足袋}	&tabi	& чевли со одвоен палец\\
{\fontspec{Sazanami Gothic}草履}	&zori	& влечки\\
{\fontspec{Sazanami Gothic}袴}	&hakama	& широки пантолоно од старојапонската носија\\
\end{tabular}
\vspace{.5cm}
}}

\def\gkakobogaku{{
\noindent\par\begin{tabular}{@{}p{2.75cm}p{5cm}p{8.25cm}@{}}
{\fontspec{Sazanami Gothic}\Large\bfseries 解剖学}	&\Large{\bfseries{kako·b\=o·gaku}}	& \Large{анатомија}\\
	&& \\
{\fontspec{Sazanami Gothic}腹}	&hara	& стомак, тежиште\\
{\fontspec{Sazanami Gothic}体}	&tai	& тело\\
{\fontspec{Sazanami Gothic}正面}	&sh\=o·men	& преден дел на главата\\
{\fontspec{Sazanami Gothic}横面}	&yoko·men	& страничен дел на главата\\
{\fontspec{Sazanami Gothic}膝}	&hiza	& колено\\
{\fontspec{Sazanami Gothic}首}	&kubi	& врат\\
{\fontspec{Sazanami Gothic}胸}	&mune	& гради\\
{\fontspec{Sazanami Gothic}肩}	&kata	& рамо\\
{\fontspec{Sazanami Gothic}肘}	&hiji	& лакт\\
{\fontspec{Sazanami Gothic}腕}	&ude	& рака\\
{\fontspec{Sazanami Gothic}手首}	&te·kubi	& зглоб\\
{\fontspec{Sazanami Gothic}手}	&te	& дланка\\
{\fontspec{Sazanami Gothic}手刀}	&te·gatana	& дланка меч\\
{\fontspec{Sazanami Gothic}足}	&ashi	& нога, стопало\\
{\fontspec{Sazanami Gothic}足首}	&asho kubi	& глужд\\
{\fontspec{Sazanami Gothic}腰}	&koshi	& колкови\\
{\fontspec{Sazanami Gothic}襟}	&eri	& крагна\\
{\fontspec{Sazanami Gothic}身}	&mi	& тело\\
{\fontspec{Sazanami Gothic}袖}	&sode	& ?\\
\end{tabular}
\vspace{.5cm}
}}

\def\g?{{
\noindent\par\begin{tabular}{@{}p{2.75cm}p{5cm}p{8.25cm}@{}}
{\fontspec{Sazanami Gothic}\Large\bfseries ?}	&\Large{\bfseries{?}}	& \Large{разно}\\
	&& \\
{\fontspec{Sazanami Gothic}合抜け}	&ai·nuke	& взаемно избегнување\\
{\fontspec{Sazanami Gothic}合打ち}	&ai·uchi	& вземен погодок\\
{\fontspec{Sazanami Gothic}当て身}	&a·te·mi	& удар\\
{\fontspec{Sazanami Gothic}武道}	&budo	& воена наука\\
{\fontspec{Sazanami Gothic}組杖}	&kumi·j\=o	& вежби со стап со партнер\\
{\fontspec{Sazanami Gothic}組太刀}	&kumi·ta·chi	& вежби со меч со партнер\\
{\fontspec{Sazanami Gothic}太刀取り}	&ta·chi·do·ri	& одземање на меч\\
{\fontspec{Sazanami Gothic}短刀取り}	&tan·to·do·ri	& одземање на нож\\
{\fontspec{Sazanami Gothic}正眼}	&sei·gan	& покажување кон окото\\
{\fontspec{Sazanami Gothic}構え}	&kama·e	& став\\
{\fontspec{Sazanami Gothic}間合い}	&ma·a·i	& ?\\
{\fontspec{Sazanami Gothic}膝行}	&shikkou	& ?\\
{\fontspec{Sazanami Gothic}太鼓}	&tai·ko	& ?\\
\end{tabular}
\vspace{.5cm}
}}

\def\g?{{
\noindent\par\begin{tabular}{@{}p{2.75cm}p{5cm}p{8.25cm}@{}}
{\fontspec{Sazanami Gothic}\Large\bfseries 方}	&\Large{\bfseries{?}}	& \Large{правци}\\
	&& \\
{\fontspec{Sazanami Gothic}左}	&hidari	& лево\\
{\fontspec{Sazanami Gothic}右}	&migi	& десно\\
{\fontspec{Sazanami Gothic}入身}	&iri·mi	& влез\\
{\fontspec{Sazanami Gothic}回転}	&kai·ten	& одзади\\
{\fontspec{Sazanami Gothic}前}	&mae	& напред\\
{\fontspec{Sazanami Gothic}後ろ}	&ushi·ro	& наназад\\
{\fontspec{Sazanami Gothic}横}	&yoko	& странично\\
{\fontspec{Sazanami Gothic}内}	&uchi	& внатрешно\\
{\fontspec{Sazanami Gothic}外}	&soto	& надворешно\\
{\fontspec{Sazanami Gothic}斜め}	&nana·me	& ?\\
{\fontspec{Sazanami Gothic}直立}	&choku·ritsu	&  [?]\\
{\fontspec{Sazanami Gothic}水平}	&sui·hei	&  [?]\\
{\fontspec{Sazanami Gothic}立て}	&ta·te	& ?\\
{\fontspec{Sazanami Gothic}反対}	&han·tai	& спротивно\\
{\fontspec{Sazanami Gothic}八方}	&hap·p\=o	& осум правци\\
\end{tabular}
\vspace{.5cm}
}}

\def\gundo{{
\noindent\par\begin{tabular}{@{}p{2.75cm}p{5cm}p{8.25cm}@{}}
{\fontspec{Sazanami Gothic}\Large\bfseries 運動}	&\Large{\bfseries{un·d\=o}}	& \Large{?}\\
	&& \\
{\fontspec{Sazanami Gothic}一教運動}	&ik·kyo undo	& ?\\
{\fontspec{Sazanami Gothic}体の変更}	&tai no hen·kou	& ?\\
\end{tabular}
\vspace{.5cm}
}}

\def\ghojonokata{{
\noindent\par\begin{tabular}{@{}p{2.75cm}p{5cm}p{8.25cm}@{}}
{\fontspec{Sazanami Gothic}\Large\bfseries 法定之形}	&\Large{\bfseries{h\=o·j\=o no kata}}	& \Large{?}\\
	&& \\
{\fontspec{Sazanami Gothic}春の太刀}	&haru no ta·chi	&  [?]\\
{\fontspec{Sazanami Gothic}夏の太刀}	&natsu no ta·chi	&  [?]\\
{\fontspec{Sazanami Gothic}秋の太刀}	&aki no ta·chi	&  [?]\\
{\fontspec{Sazanami Gothic}冬の太刀}	&fuyu no ta·chi	&  [?]\\
{\fontspec{Sazanami Gothic}八相発破}	&hass\=o happa	& ?\\
{\fontspec{Sazanami Gothic}一刀両断}	&itto ry\=o·dan	& ?\\
{\fontspec{Sazanami Gothic}右転左転}	&u·ten sa·ten	& ?\\
{\fontspec{Sazanami Gothic}長短一身}	&cho·tan ichi·mi	& ?\\
{\fontspec{Sazanami Gothic}受ける }	&u·ke·ru	&  [?]\\
{\fontspec{Sazanami Gothic}体剣}	&tai·ken	&  [?]\\
{\fontspec{Sazanami Gothic}目礼}	&moku·rei	& ?\\
{\fontspec{Sazanami Gothic}抜剣}	&bak·ken	&  [?]\\
{\fontspec{Sazanami Gothic}合?}	&ai seigan	& ?\\
{\fontspec{Sazanami Gothic}促進}	&soku·shin	&  [?]\\
{\fontspec{Sazanami Gothic}仁王太刀}	&ni·\=o da·chi	&  [?]\\
{\fontspec{Sazanami Gothic}?}	&kazashi	&  [?]\\
{\fontspec{Sazanami Gothic}合上段}	&ai j\=o·dan	&  [?]\\
{\fontspec{Sazanami Gothic}?}	&tsume	&  [?]\\
{\fontspec{Sazanami Gothic}打ち込み}	&u·chi ko·mi	& ?\\
{\fontspec{Sazanami Gothic}両腕}	&mor\=o·de	& ?\\
{\fontspec{Sazanami Gothic}押し込み}	&o·shi ko·mi	& ?\\
{\fontspec{Sazanami Gothic}一文字}	&ichi·mon·ji	& ?\\
{\fontspec{Sazanami Gothic}?}	&so tai	&  [?]\\
{\fontspec{Sazanami Gothic}体当り}	&tai ata·ri	& ?\\
{\fontspec{Sazanami Gothic}?}	&hi tachi	& ?\\
{\fontspec{Sazanami Gothic}霞}	&kasumi	& ?\\
{\fontspec{Sazanami Gothic}一重み}	&hi·toe·mi	& ?\\
{\fontspec{Sazanami Gothic}?}	&muki	&  [?]\\
{\fontspec{Sazanami Gothic}切り}	&ki·ri	&  [?]\\
{\fontspec{Sazanami Gothic}止め}	&to·me	&  [?]\\
{\fontspec{Sazanami Gothic}裏打ち}	&ura u·chi	&  [?]\\
{\fontspec{Sazanami Gothic}曙光}	&sho·k\=o	&  [?]\\
{\fontspec{Sazanami Gothic}大}	&dai	&  [големи]\\
{\fontspec{Sazanami Gothic}退けん}	&no·ke·n	&  [?]\\
\end{tabular}
\vspace{.5cm}
}}

\def\ggenkikai{{
\noindent\par\begin{tabular}{@{}p{2.75cm}p{5cm}p{8.25cm}@{}}
{\fontspec{Sazanami Gothic}\Large\bfseries 元気会}	&\Large{\bfseries{gen·ki·kai}}	& \Large{?}\\
	&& \\
{\fontspec{Sazanami Gothic}大円呼吸法}	&dai en ko·ky\=u h\=o	& дишење во големи кругови [?]\\
{\fontspec{Sazanami Gothic}守有の呼吸}	&su·u no ko·ky\=u	& основна вежба за дишење [?]\\
{\fontspec{Sazanami Gothic}陽の手呼吸}	&yo no te ko·ky\=u	& дишење со рацете во јанг позиција [?]\\
{\fontspec{Sazanami Gothic}陰の手呼吸}	&in no te ko·ky\=u	& дишење со рацете во јин позиција [?]\\
{\fontspec{Sazanami Gothic}気結びの手呼吸}	&ki·musu·bi no te ko·ky\=u	& дишење со рацете формирајќи крстесто движење [?]\\
{\fontspec{Sazanami Gothic}阿吽の呼吸}	&a·un no ko·ky\=u	& дишење 'да се стане едно со вселената' [?]\\
{\fontspec{Sazanami Gothic}元の呼吸}	&gen no ko·ky\=u	&  [?]\\
{\fontspec{Sazanami Gothic}寝運動}	&ne un·d\=o	&  [?]\\
{\fontspec{Sazanami Gothic}揺動法}	&y\=o d\=o·h\=o	&  [?]\\
{\fontspec{Sazanami Gothic}毛管運動}	&m\=o·kan un·d\=o	&  [?]\\
{\fontspec{Sazanami Gothic}合掌合蹠運動}	&gas·sh\=o gas·seki un·d\=o	&  [?]\\
{\fontspec{Sazanami Gothic}金魚運動}	&kin·gyo un·d\=o	&  [?]\\
{\fontspec{Sazanami Gothic}馬運動}	&uma un·d\=o	&  [?]\\
\end{tabular}
\vspace{.5cm}
}}

