\def\gryu{{
\noindent\par\begin{tabular}{@{}p{2.75cm}p{5cm}p{8.25cm}@{}}
{\fontspec{Sazanami Gothic}\Large\bfseries 竜}	&\Large{\bfseries{ry\=u}}	& \Large{type}\\
	&& \\
{\fontspec{Sazanami Gothic}合気道}	&ai·ki·d\=o	& weg van harmoniseren van ki\\
{\fontspec{Sazanami Gothic}合気?}	&ai·ki·jutsu	& lichaamstechnieken die aiki-principe gebruiken\\
{\fontspec{Sazanami Gothic}合気杖}	&ai·ki·j\=o	& j\=o technieken die aiki-principe gebruiken\\
{\fontspec{Sazanami Gothic}合気剣}	&ai·ki·ken	& ken technieken die aiki-principe gebruiken\\
{\fontspec{Sazanami Gothic}法定形}	&ho·jo kata	& eerste kata van specifieke zwaardstijl\\
{\fontspec{Sazanami Gothic}鹿島神傳直心影流}	&Kashima Shinden Jikishinkage-ry\=u	& zwaardstijl\\
{\fontspec{Sazanami Gothic}元気会}	&gen·ki·kai	& groep van gezondheisoefeningen\\
{\fontspec{Sazanami Gothic}整体}	&sei·tai	& correct geordend lichaam\\
\end{tabular}
\vspace{.5cm}
}}

\def\gkeikoho{{
\noindent\par\begin{tabular}{@{}p{2.75cm}p{5cm}p{8.25cm}@{}}
{\fontspec{Sazanami Gothic}\Large\bfseries ?}	&\Large{\bfseries{keiko h\=o}}	& \Large{TODOgeest}\\
	&& \\
{\fontspec{Sazanami Gothic}?稽古}	&kakari kei·ko	& één voor één aanvallen [continu oefening]\\
{\fontspec{Sazanami Gothic}?}	&kigata	& vloeiende aanvallen, tori laat zich niet vastpakken [energetische vorm]\\
{\fontspec{Sazanami Gothic}鍛錬}	&tan·ren	& gedisciplineerde training, tori laat zich vastpakken [gedisciplineerd]\\
{\fontspec{Sazanami Gothic}真剣}	&shin·ken	& realistische training [serieus]\\
{\fontspec{Sazanami Gothic}崩し}	&kuzushi	& balansverstoring\\
{\fontspec{Sazanami Gothic}後ろ技}	&ushi·ro·waza	& technieken voor aanvallen van achter\\
{\fontspec{Sazanami Gothic}自由技}	&ji·y\=u waza	& vrije vorm techniek\\
{\fontspec{Sazanami Gothic}乱取り}	&ran·do·ri	& technieken met meerdere aanvallers\\
{\fontspec{Sazanami Gothic}素振り}	&su·bu·ri	&  [herhaaldeijke oefening]\\
\end{tabular}
\vspace{.5cm}
}}

\def\g?{{
\noindent\par\begin{tabular}{@{}p{2.75cm}p{5cm}p{8.25cm}@{}}
{\fontspec{Sazanami Gothic}\Large\bfseries ?}	&\Large{\bfseries{?}}	& \Large{positie}\\
	&& \\
{\fontspec{Sazanami Gothic}座り技}	&suwa·ri waza	& zittende techniek\\
{\fontspec{Sazanami Gothic}半身半立ち技}	&han·mi han·da·chi waza	& techniek voor zittende tori en staande uke\\
{\fontspec{Sazanami Gothic}立ち技}	&ta·chi waza	& staande techniek\\
{\fontspec{Sazanami Gothic}座}	&za	& zittend\\
{\fontspec{Sazanami Gothic}正座}	&sei·za	& op knieën zittend met voeten plat op grond\\
{\fontspec{Sazanami Gothic}跪座}	&ki·za	& op knieën zittend met tenen in mat\\
{\fontspec{Sazanami Gothic}割座}	&wari·za	& op knieën zittend met heupen tussen voeten op grond\\
{\fontspec{Sazanami Gothic}?座}	&cho·za	& zittend met benen vooruitgestrekt\\
{\fontspec{Sazanami Gothic}?座}	&kai·za	& zittend met benen gestrekt en geopend\\
{\fontspec{Sazanami Gothic}体座}	&tai·za	& zittend met knieën tegen borst en voeten op grond\\
{\fontspec{Sazanami Gothic}?座}	&gaku·za	& zittend met voetzolen tegen elkaar en knieën op grond\\
{\fontspec{Sazanami Gothic}安座}	&an·za	& kleermakerszit zonder benen te kruisen\\
{\fontspec{Sazanami Gothic}八相(八双)}	&hass\=o	& positie als een acht [hass\=ogamae]\\
{\fontspec{Sazanami Gothic}?座}	&kahuza	& lotuspositie\\
\end{tabular}
\vspace{.5cm}
}}

\def\gaite{{
\noindent\par\begin{tabular}{@{}p{2.75cm}p{5cm}p{8.25cm}@{}}
{\fontspec{Sazanami Gothic}\Large\bfseries 相手}	&\Large{\bfseries{ai·te}}	& \Large{partner}\\
	&& \\
{\fontspec{Sazanami Gothic}多人数取り}	&ta·nin·z\=u to·ri	& drie of meer uke die continue aanvallen, alleen begin techniek doen\\
{\fontspec{Sazanami Gothic}掛り稽古}	&ka·ka·ri gei·ko	& twee of meer uke die om de beurt aanvallen\\
\end{tabular}
\vspace{.5cm}
}}

\def\gbuki{{
\noindent\par\begin{tabular}{@{}p{2.75cm}p{5cm}p{8.25cm}@{}}
{\fontspec{Sazanami Gothic}\Large\bfseries 武器}	&\Large{\bfseries{bu·ki}}	& \Large{wapen}\\
	&& \\
{\fontspec{Sazanami Gothic}杖}	&j\=o	& stok of korte houten staf\\
{\fontspec{Sazanami Gothic}棒}	&b\=o	& lange houten staf\\
{\fontspec{Sazanami Gothic}木剣}	&bok·ken, boku·t\=o	& gebogen houten zwaard\\
{\fontspec{Sazanami Gothic}短刀}	&tan·t\=o	& mes of dolk (kling korter dan 30 cm)\\
{\fontspec{Sazanami Gothic}脇差}	&waki·zashi, shoto	& gebogen metalen kort zwaard (kling 30 - 60 cm)\\
{\fontspec{Sazanami Gothic}刀}	&(ken?,) katana	& gebogen metalen zwaard (kling 60 - 70 cm)\\
{\fontspec{Sazanami Gothic}太刀}	&ta·chi	& gebogen metalen lang zwaard (kling langer dan 70 cm)\\
{\fontspec{Sazanami Gothic}剣}	&ken	& kort recht metalen tweesnijdend zwaard\\
{\fontspec{Sazanami Gothic}日本刀}	&ni·hon·t\=o	& Japans zwaard\\
{\fontspec{Sazanami Gothic}薙刀}	&naginata	& hellebaard\\
{\fontspec{Sazanami Gothic}槍}	&yari	& speer\\
{\fontspec{Sazanami Gothic}鍔}	&tsuba	& stootplaat\\
{\fontspec{Sazanami Gothic}鞘}	&saya	& schede\\
{\fontspec{Sazanami Gothic}?}	&bugukake	& wapenrek\\
\end{tabular}
\vspace{.5cm}
}}

\def\gkokukegi{{
\noindent\par\begin{tabular}{@{}p{2.75cm}p{5cm}p{8.25cm}@{}}
{\fontspec{Sazanami Gothic}\Large\bfseries 迫撃}	&\Large{\bfseries{koku·kegi}}	& \Large{aanval}\\
	&& \\
{\fontspec{Sazanami Gothic}気合せ正面打ち}	&ki·a·wa·se sh\=o·men u·chi	& directe slag met open hand vanuit heupen naar voorkant hoofd [energetisch ontmoeten voorkant-gezicht slag]\\
{\fontspec{Sazanami Gothic}片手取り相半身}	&kata·te do·ri ai han·mi	& greep naar één pols aan zelfde kant (identiek) [één-kant-hand greep zelfde kant]\\
{\fontspec{Sazanami Gothic}片手取り逆半身}	&kata·te do·ri gyaku han·mi	& greep naar één pols aan andere kant (spiegel) [één-kant-hand greep andere kant]\\
{\fontspec{Sazanami Gothic}両手取り}	&ry\=o·te do·ri	& beide polsen grijpen [beide-polsen grijpen]\\
{\fontspec{Sazanami Gothic}正面打ち}	&sh\=o·men u·chi	& slag tegen voorkant hoofd [voorkant-gezicht slag]\\
{\fontspec{Sazanami Gothic}片手両手取り}	&kata·te ry\=o·te do·ri	& greep naar één pols met beide handen [één-kant-hand beide-handen greep]\\
{\fontspec{Sazanami Gothic}肩取り}	&kata do·ri	& greep naar één schouder [schouder greep]\\
{\fontspec{Sazanami Gothic}後ろ両手取り}	&ushi·ro ry\=o·te do·ri	& greep naar beide polsen van achter [achter beide-handen greep]\\
{\fontspec{Sazanami Gothic}横面打ち}	&yoko·men u·chi	& slag tegen zijkant hoofd [zijkant-gezicht slag]\\
{\fontspec{Sazanami Gothic}中段突き}	&ch\=u·dan tsu·ki	& stoot of steek naar middelste deel lichaam [midden lichaam steek]\\
{\fontspec{Sazanami Gothic}肩取り面打ち}	&kata do·ri men u·chi	& greep naar schouder en slag tegen voorkant hoofd [schouder greep hoofd slag]\\
{\fontspec{Sazanami Gothic}後ろ両肩取り}	&ushi·ro ry\=o·kata do·ri	& greep naar beide schouders van achter [achter beide-schouders greep]\\
{\fontspec{Sazanami Gothic}上段突き}	&j\=o·dan tsu·ki	& stoot of steek naar bovenste deel lichaam [boven-lichaam steek]\\
{\fontspec{Sazanami Gothic}両襟取り}	&ry\=o·eri do·ri	& ssk newsletter 3XXX [beide-kragen greep]\\
{\fontspec{Sazanami Gothic}蹴り}	&ke·ri	& trap\\
{\fontspec{Sazanami Gothic}後ろ首絞め}	&ushi·ro kubi jime	& nekverwurging van achter [achter nek wurg]\\
\end{tabular}
\vspace{.5cm}
}}

\def\guchi/soto{{
\noindent\par\begin{tabular}{@{}p{2.75cm}p{5cm}p{8.25cm}@{}}
{\fontspec{Sazanami Gothic}\Large\bfseries 内/外}	&\Large{\bfseries{uchi/soto}}	& \Large{binnen/buiten}\\
	&& \\
{\fontspec{Sazanami Gothic}内捌き}	&uchi saba·ki	& beweging naar binnenkant\\
{\fontspec{Sazanami Gothic}外捌き}	&soto saba·ki	& beweging naar buitenkant\\
\end{tabular}
\vspace{.5cm}
}}

\def\gtaisabaki{{
\noindent\par\begin{tabular}{@{}p{2.75cm}p{5cm}p{8.25cm}@{}}
{\fontspec{Sazanami Gothic}\Large\bfseries 体捌き}	&\Large{\bfseries{tai saba·ki}}	& \Large{lichaamsverplaatsing}\\
	&& \\
{\fontspec{Sazanami Gothic}前足入身}	&mae ashi iri·mi	& ingaan met voorste voet\\
{\fontspec{Sazanami Gothic}後ろ足入身}	&ushi·ro ashi iri·mi	& ingaan met achterste voet\\
{\fontspec{Sazanami Gothic}前足転換}	&mae ashi ten·kan	& wegdraaien op voorste voet\\
{\fontspec{Sazanami Gothic}後ろ足転換}	&ushi·ro ashi ten·kan	& wegdraaien op achterste voet\\
{\fontspec{Sazanami Gothic}前足?}	&mae ashi tenshin	& uitwijken met voorste voet\\
{\fontspec{Sazanami Gothic}後ろ足?}	&ushi·ro ashi tenshin	& uitwijken met achterste voet\\
{\fontspec{Sazanami Gothic}前足入身転換}	&mae ashi iri·mi ten·kan	& ingaan met voorste voet en daarmee wegdraaien\\
{\fontspec{Sazanami Gothic}後ろ足入身転換}	&ushi·ro ashi iri·mi ten·kan	& ingaan met achterste voet en daarmee wegdraaien\\
{\fontspec{Sazanami Gothic}前足入身転換?}	&mae ashi iri·mi ten·kan tenshin	& ingaan met voorste voet, daarmee wegdraaien en uitwijken\\
{\fontspec{Sazanami Gothic}後ろ足入身転換?}	&ushi·ro ashi iri·mi ten·kan tenshin	& ingaan met achterste voet, daarmee wegdraaien en uitwijken\\
{\fontspec{Sazanami Gothic}前足転換?}	&mae ashi ten·kan tenshin	& wegdraaien op voorste voet en uitwijken\\
{\fontspec{Sazanami Gothic}後ろ足転換?}	&ushi·ro ashi ten·kan tenshin	& wegdraaien op achterste voet en uitwijken\\
{\fontspec{Sazanami Gothic}後ろ足入身?}	&ushi·ro ashi iri·mi tenkai	& ingaan met voorste voet en hele lichaam op de plek draaien\\
\end{tabular}
\vspace{.5cm}
}}

\def\gtesabaki{{
\noindent\par\begin{tabular}{@{}p{2.75cm}p{5cm}p{8.25cm}@{}}
{\fontspec{Sazanami Gothic}\Large\bfseries 手捌き}	&\Large{\bfseries{te saba·ki}}	& \Large{handbeweging}\\
	&& \\
{\fontspec{Sazanami Gothic}上半円}	&kami han·en	& halve cirkel bovenlangs\\
{\fontspec{Sazanami Gothic}下半円}	&shimo han·en	& cirkel onderlangs\\
{\fontspec{Sazanami Gothic}手首返し}	&te·kubi gae·shi	& onderarmen parallel\\
{\fontspec{Sazanami Gothic}十字結び}	&jy\=u·ji musu·bi	& onderarmen gekruisd\\
{\fontspec{Sazanami Gothic}受け流し}	&uke naga·shi	& ?\\
\end{tabular}
\vspace{.5cm}
}}

\def\gnote{{
\noindent\par\begin{tabular}{@{}p{2.75cm}p{5cm}p{8.25cm}@{}}
{\fontspec{Sazanami Gothic}\Large\bfseries の手}	&\Large{\bfseries{no·te}}	& \Large{hand}\\
	&& \\
{\fontspec{Sazanami Gothic}打ちの手}	&u·chi·no·te	& slaghand\\
{\fontspec{Sazanami Gothic}肩の手}	&kata·no·te	& hand bij schouder\\
\end{tabular}
\vspace{.5cm}
}}

\def\gdan{{
\noindent\par\begin{tabular}{@{}p{2.75cm}p{5cm}p{8.25cm}@{}}
{\fontspec{Sazanami Gothic}\Large\bfseries 段}	&\Large{\bfseries{dan}}	& \Large{hoogte}\\
	&& \\
{\fontspec{Sazanami Gothic}下段}	&ge·dan	& laag\\
{\fontspec{Sazanami Gothic}中段}	&ch\=u·dan	& midden\\
{\fontspec{Sazanami Gothic}上段}	&j\=o·dan	& hoog\\
{\fontspec{Sazanami Gothic}脇構}	&waki gamae	& kleine houding\\
{\fontspec{Sazanami Gothic}下段の構え}	&ge·dan no kama·e	& lage houding\\
{\fontspec{Sazanami Gothic}中段の構え}	&ch\=u·dan no kama·e	& midden houden\\
{\fontspec{Sazanami Gothic}上段の構}	&j\=o·dan no kamae	& hoge houding\\
\end{tabular}
\vspace{.5cm}
}}

\def\gwaza{{
\noindent\par\begin{tabular}{@{}p{2.75cm}p{5cm}p{8.25cm}@{}}
{\fontspec{Sazanami Gothic}\Large\bfseries 技}	&\Large{\bfseries{waza}}	& \Large{techniek}\\
	&& \\
{\fontspec{Sazanami Gothic}一教}	&ik·ky\=o	&  [eerste onderricht]\\
{\fontspec{Sazanami Gothic}二教}	&ni·ky\=o	&  [tweede onderricht]\\
{\fontspec{Sazanami Gothic}三教}	&san·ky\=o	&  [derde onderricht]\\
{\fontspec{Sazanami Gothic}四教}	&yon·ky\=o	&  [vierde onderricht]\\
{\fontspec{Sazanami Gothic}隅落し}	&sumi·oto·shi	& hoek worp\\
{\fontspec{Sazanami Gothic}小手返し}	&ko·te·gae·shi	& pols verdraaiïng\\
{\fontspec{Sazanami Gothic}入身投げ}	&iri·mi·na·ge	& ingaande worp\\
{\fontspec{Sazanami Gothic}四方投げ}	&shi·h\=o·na·ge	& vier richtingen worp\\
{\fontspec{Sazanami Gothic}五教}	&go·ky\=o	& vijfde onderricht\\
{\fontspec{Sazanami Gothic}肘決め抑え}	&hiji ki·me osa·e	& elleboog controle klem\\
{\fontspec{Sazanami Gothic}内回転三教}	&uchi·kai·ten·san·ky\=o	& binnenlangs raddraai sankyo\\
{\fontspec{Sazanami Gothic}腕絡み}	&ude gara·mi	& arm verstrengeling\\
{\fontspec{Sazanami Gothic}合気腰}	&ai·ki·goshi	& heupworp met hoofd van binnen naar buiten onderlangs gyaku hanmi\\
{\fontspec{Sazanami Gothic}合気落し}	&ai·ki·oto·shi	&  [vermengend laten vallen]\\
{\fontspec{Sazanami Gothic}回転投げ}	&kai·ten·na·ge	&  [raddraai worp]\\
{\fontspec{Sazanami Gothic}腕決め投げ}	&ude ki·me na·ge	&  [arm breek worp]\\
{\fontspec{Sazanami Gothic}前落し}	&mae·oto·shi	&  [voorwaarste worp]\\
{\fontspec{Sazanami Gothic}引き落し}	&hiki·oto·shi	&  [trek worp]\\
{\fontspec{Sazanami Gothic}切り落し}	&ki·ri oto·shi	&  [houw worp]\\
{\fontspec{Sazanami Gothic}回転落し}	&kai·ten·oto·shi	& draai worp\\
{\fontspec{Sazanami Gothic}天地投げ}	&ten chi na·ge	& hemel aarde worp\\
{\fontspec{Sazanami Gothic}玄形呼吸投げ}	&gen·kei·ko·ky\=u·na·ge	& archetype ademworp\\
{\fontspec{Sazanami Gothic}内回転投げ}	&uchi·kai·ten·na·ge	& binnenlangs raddraai worp\\
{\fontspec{Sazanami Gothic}振り突き呼吸投げ}	&furi·zu·ki ko·ky\=u na·ge	& slingerende stomp ademworp\\
{\fontspec{Sazanami Gothic}十字絡み}	&jy\=u·ji gara·mi	&  [kruis knoop verstrengeling]\\
{\fontspec{Sazanami Gothic}鳥り船呼吸投げ}	&to·ri fune ko·ky\=u·na·ge	&  [roeioefening ademworp]\\
{\fontspec{Sazanami Gothic}四方蹴り呼吸投げ}	&shi·h\=o ge·ri ko·ky\=u·na·ge	&  [vier richtingen trappen ademworp]\\
{\fontspec{Sazanami Gothic}教絡み三教投げ}	&ude gara·mi san·ky\=o na·ge	& arm verstrengeling sankyo worp\\
{\fontspec{Sazanami Gothic}教絡み四教投げ}	&ude gara·mi yon·ky\=o na·ge	& arm verstrengeling yonkyo worp\\
{\fontspec{Sazanami Gothic}腰投げ}	&koshi na·ge	& heupworp\\
{\fontspec{Sazanami Gothic}外回転投げ}	&soto kai·ten·na·ge	& buitenlangs raddraai worp\\
{\fontspec{Sazanami Gothic}斬刀呼吸投げ}	&zan·to ko·ky\=u na·ge	& ?\\
{\fontspec{Sazanami Gothic}教絡み抑え}	&ude gara·mi osa·e	& arm verstrengeling klem\\
{\fontspec{Sazanami Gothic}手車}	&te guruma	& hand rad\\
{\fontspec{Sazanami Gothic}背負い車}	&se·o·i guruma	& schouder rad\\
{\fontspec{Sazanami Gothic}腰車}	&koshi guruma	& heup ?\\
{\fontspec{Sazanami Gothic}沈身腰車}	&chin shin koshi guruma	& zinkend lichaam heup wiel\\
\end{tabular}
\vspace{.5cm}
}}

\def\gomote/ura{{
\noindent\par\begin{tabular}{@{}p{2.75cm}p{5cm}p{8.25cm}@{}}
{\fontspec{Sazanami Gothic}\Large\bfseries 表/裏}	&\Large{\bfseries{omote/ura}}	& \Large{voor/achter}\\
	&& \\
{\fontspec{Sazanami Gothic}表}	&omote	& voor partner\\
{\fontspec{Sazanami Gothic}裏}	&ura	& achter partner\\
\end{tabular}
\vspace{.5cm}
}}

\def\gyin/yang{{
\noindent\par\begin{tabular}{@{}p{2.75cm}p{5cm}p{8.25cm}@{}}
{\fontspec{Sazanami Gothic}\Large\bfseries ?/?}	&\Large{\bfseries{yin/yang}}	& \Large{yin/yang}\\
	&& \\
{\fontspec{Sazanami Gothic}?}	&yin	& yin\\
{\fontspec{Sazanami Gothic}?}	&yang	& yang\\
\end{tabular}
\vspace{.5cm}
}}

\def\gri{{
\noindent\par\begin{tabular}{@{}p{2.75cm}p{5cm}p{8.25cm}@{}}
{\fontspec{Sazanami Gothic}\Large\bfseries 理}	&\Large{\bfseries{ri}}	& \Large{principe}\\
	&& \\
{\fontspec{Sazanami Gothic}水}	&su, mizu	& water, boven-onder, oost\\
{\fontspec{Sazanami Gothic}土}	&do, tsu	& aarde, links-rechts, zuid\\
{\fontspec{Sazanami Gothic}風}	&hu	& wind, voor-achter, west\\
{\fontspec{Sazanami Gothic}火}	&ka, hi	& vuur, spiraal, noord\\
{\fontspec{Sazanami Gothic}?}	&complete?	& mens\\
{\fontspec{Sazanami Gothic}春}	&haru	& lente\\
{\fontspec{Sazanami Gothic}夏}	&natsu	& zomer\\
{\fontspec{Sazanami Gothic}秋}	&aki	& herfst\\
{\fontspec{Sazanami Gothic}冬}	&fuyu	& winter\\
{\fontspec{Sazanami Gothic}攻防の原理}	&k\=o·b\=o no gen·ri	& theorie van aanval en verdediging\\
{\fontspec{Sazanami Gothic}打ちの理}	&u·chi no ri	& stoot principe\\
{\fontspec{Sazanami Gothic}抑えの理}	&osa·e no ri	& klem principe\\
{\fontspec{Sazanami Gothic}投げの理}	&na·ge no ri	& werp principe\\
{\fontspec{Sazanami Gothic}斬の理}	&zan no ri	& snij principe\\
\end{tabular}
\vspace{.5cm}
}}

\def\gnage/osae{{
\noindent\par\begin{tabular}{@{}p{2.75cm}p{5cm}p{8.25cm}@{}}
{\fontspec{Sazanami Gothic}\Large\bfseries 投げ/抑え}	&\Large{\bfseries{nage/osae}}	& \Large{nage/osae}\\
	&& \\
{\fontspec{Sazanami Gothic}投げ}	&na·ge	& worp\\
{\fontspec{Sazanami Gothic}抑え}	&osa·e	& klem\\
{\fontspec{Sazanami Gothic}投げ抑え}	&na·ge osa·e	& geklemd werpen\\
\end{tabular}
\vspace{.5cm}
}}

\def\gukemi{{
\noindent\par\begin{tabular}{@{}p{2.75cm}p{5cm}p{8.25cm}@{}}
{\fontspec{Sazanami Gothic}\Large\bfseries 受身}	&\Large{\bfseries{u·ke·mi}}	& \Large{ontvangen met het lichaam}\\
	&& \\
	&boven onder u·ke·mi	& boven onder ukemi\\
{\fontspec{Sazanami Gothic}前受身}	&mae u·ke·mi	& voorwaartse ukemi\\
{\fontspec{Sazanami Gothic}後ろ受身}	&ushi·ro u·ke·mi	& achterwaartse ukemi\\
{\fontspec{Sazanami Gothic}横受身}	&yoko u·ke·mi	& zijwaartse ukemi\\
	&chokuto	& valbreken door op de plaats af te slaan\\
{\fontspec{Sazanami Gothic}飛び受身}	&to·bi u·ke·mi	& vrije val\\
\end{tabular}
\vspace{.5cm}
}}

\def\g?{{
\noindent\par\begin{tabular}{@{}p{2.75cm}p{5cm}p{8.25cm}@{}}
{\fontspec{Sazanami Gothic}\Large\bfseries ?}	&\Large{\bfseries{?}}	& \Large{persoon}\\
	&& \\
{\fontspec{Sazanami Gothic}植芝 盛平}	&Ueshiba, Morihei	& wijlen oprichter van aikid\=o, \=osensei (14-12-1883 - 26-04-1969)\\
{\fontspec{Sazanami Gothic}植芝 吉祥丸}	&Ueshiba, Kisshomaru	& wijlen zoon van \=osensei, tweede d\=oshu (27-06-1921 - 04-01-1999)\\
{\fontspec{Sazanami Gothic}植芝 守央}	&Moriteru Ueshiba	& kleinzoon van \=osensei, derde d\=oshu (02-04-1951)\\
{\fontspec{Sazanami Gothic}多田 宏}	&Tada, Hiroshi	& leerling van \=osensei, shihan (13-12-1929)\\
{\fontspec{Sazanami Gothic}池田 昌富}	&Ikeda, Masatomi	& leerling van Tada sensei, shihan (08-04-1940)\\
\end{tabular}
\vspace{.5cm}
}}

\def\gkazueru{{
\noindent\par\begin{tabular}{@{}p{2.75cm}p{5cm}p{8.25cm}@{}}
{\fontspec{Sazanami Gothic}\Large\bfseries 数える}	&\Large{\bfseries{kazu·e·ru}}	& \Large{tellen}\\
	&& \\
{\fontspec{Sazanami Gothic}○}	&zero	& nul, leeg\\
{\fontspec{Sazanami Gothic}一}	&ichi	& een\\
{\fontspec{Sazanami Gothic}二}	&ni	& twee\\
{\fontspec{Sazanami Gothic}三}	&san	& drie\\
{\fontspec{Sazanami Gothic}四}	&shi, yon	& vier\\
{\fontspec{Sazanami Gothic}五}	&go	& vijf\\
{\fontspec{Sazanami Gothic}六}	&roku	& zes\\
{\fontspec{Sazanami Gothic}七}	&shichi, nana	& zeven\\
{\fontspec{Sazanami Gothic}八}	&hachi	& acht\\
{\fontspec{Sazanami Gothic}九}	&ku, ky\=u	& negen\\
{\fontspec{Sazanami Gothic}十}	&j\=u	& tien\\
{\fontspec{Sazanami Gothic}十一}	&j\=u ichi	& elf\\
{\fontspec{Sazanami Gothic}二十}	&ni j\=u	& twintig\\
{\fontspec{Sazanami Gothic}二十一}	&ni j\=u ichi	& eenentwintig\\
{\fontspec{Sazanami Gothic}百}	&hyaku	& honderd\\
{\fontspec{Sazanami Gothic}千}	&sen	& duizend\\
{\fontspec{Sazanami Gothic}万}	&man	& tienduizend\\
{\fontspec{Sazanami Gothic}一本目}	&ip·pon·me	& eerste deel\\
{\fontspec{Sazanami Gothic}ニ本目}	&ni·hon·me	& tweede deel\\
{\fontspec{Sazanami Gothic}三本目}	&san·bon·me	& derde deel\\
{\fontspec{Sazanami Gothic}四本目}	&yon·hon·me	& vierde deel\\
{\fontspec{Sazanami Gothic}五本目}	&go·hon·me	& vijfde deel\\
{\fontspec{Sazanami Gothic}六本目}	&roku·hon·me	& zesde deel\\
{\fontspec{Sazanami Gothic}七本目}	&nana·hon·me	& zevende deel\\
{\fontspec{Sazanami Gothic}八本目}	&hachi·hon·me	& achtste deel\\
\end{tabular}
\vspace{.5cm}
}}

\def\gdankai{{
\noindent\par\begin{tabular}{@{}p{2.75cm}p{5cm}p{8.25cm}@{}}
{\fontspec{Sazanami Gothic}\Large\bfseries 段階}	&\Large{\bfseries{dan·kai}}	& \Large{gradaties}\\
	&& \\
{\fontspec{Sazanami Gothic}段}	&dan	& dangraad\\
{\fontspec{Sazanami Gothic}級}	&ky\=u	& ky\=ugraad\\
{\fontspec{Sazanami Gothic}無級}	&mu·ky\=u	& zonder ky\=u\\
{\fontspec{Sazanami Gothic}六級}	&rok·ky\=u	& zesde ky\=u\\
{\fontspec{Sazanami Gothic}五級}	&go·ky\=u	& vijfde ky\=u\\
{\fontspec{Sazanami Gothic}四級}	&yon·ky\=u	& vierde ky\=u\\
{\fontspec{Sazanami Gothic}参級}	&san·ky\=u	& derde ky\=u\\
{\fontspec{Sazanami Gothic}弐級}	&ni·ky\=u	& tweede ky\=u\\
{\fontspec{Sazanami Gothic}壱級}	&ik·ky\=u	& eerste ky\=u, beginnersky\=u\\
{\fontspec{Sazanami Gothic}有段者}	&y\=u·dan·sha	& met dan\\
{\fontspec{Sazanami Gothic}無段者}	&mu·dan·sha	& zonder dan\\
{\fontspec{Sazanami Gothic}初段}	&sho·dan	& eerste dan, beginnersdan\\
{\fontspec{Sazanami Gothic}弐段}	&ni·dan	& tweede dan\\
{\fontspec{Sazanami Gothic}参段}	&san·dan	& derde dan\\
{\fontspec{Sazanami Gothic}四段}	&yon·dan	& vierde dan\\
{\fontspec{Sazanami Gothic}五段}	&go·dan	& vijfde dan\\
{\fontspec{Sazanami Gothic}六段}	&roku·dan	& zesde dan\\
{\fontspec{Sazanami Gothic}七段}	&nana·dan	& zevende dan\\
{\fontspec{Sazanami Gothic}八段}	&hachi·dan	& achtste dan\\
{\fontspec{Sazanami Gothic}九段}	&ku·dan	& negende dan\\
{\fontspec{Sazanami Gothic}十段}	&jy\=u·dan	& tiende dan\\
\end{tabular}
\vspace{.5cm}
}}

\def\g?{{
\noindent\par\begin{tabular}{@{}p{2.75cm}p{5cm}p{8.25cm}@{}}
{\fontspec{Sazanami Gothic}\Large\bfseries ?}	&\Large{\bfseries{?}}	& \Large{algemeen}\\
	&& \\
{\fontspec{Sazanami Gothic}合}	&ai	& harmonie, eenwording\\
{\fontspec{Sazanami Gothic}気}	&ki	& levenskracht, levensenergie\\
{\fontspec{Sazanami Gothic}道}	&d\=o	& weg, leer, discipline\\
{\fontspec{Sazanami Gothic}合気道}	&ai·ki·d\=o	& weg van harmoniseren van ki\\
{\fontspec{Sazanami Gothic}合気道家}	&ai·ki·d\=o·ka	& beoefenaar van aikido\\
{\fontspec{Sazanami Gothic}合気会}	&ai·ki·kai	& organisatie voor de beoefening en verspreiding van aikid\=o\\
{\fontspec{Sazanami Gothic}受け}	&u·ke	& aanvaller [ontvanger]\\
{\fontspec{Sazanami Gothic}取り}	&to·ri	& verdediger [pakken, opnemen, kiezen]\\
{\fontspec{Sazanami Gothic}投げ}	&na·ge	&  [werper]\\
{\fontspec{Sazanami Gothic}打太刀}	&uchi·da·chi	& ouder, leraar [slaand zwaard]\\
{\fontspec{Sazanami Gothic}受太刀}	&shi·da·chi	& kind, leerling [uitvoerend zwaard]\\
\end{tabular}
\vspace{.5cm}
}}

\def\g?{{
\noindent\par\begin{tabular}{@{}p{2.75cm}p{5cm}p{8.25cm}@{}}
{\fontspec{Sazanami Gothic}\Large\bfseries ?}	&\Large{\bfseries{?}}	& \Large{instructies}\\
	&& \\
{\fontspec{Sazanami Gothic}お願いします}	&o·nega·i shi·ma·su	& wij groeten u beleeft ?/ alstublieft (aan begin van les of oefening)\\
{\fontspec{Sazanami Gothic}どうも有難う御座いました}	&do·o·mo a·ri·ga·to·o go·za·i·ma·shi·ta	& dank u wel (aan einde van les of oefening)\\
{\fontspec{Sazanami Gothic}はい}	&hai	& ja\\
{\fontspec{Sazanami Gothic}止め}	&yame	& stop (en ga aan de kant zitten)\\
{\fontspec{Sazanami Gothic}始め}	&haji·me	& begin, start\\
{\fontspec{Sazanami Gothic}待って}	&ma·t·te	& wacht, stop\\
{\fontspec{Sazanami Gothic}立て}	&ta·te	& sta op\\
{\fontspec{Sazanami Gothic}?}	&suwatte	& ga zitten\\
{\fontspec{Sazanami Gothic}?}	&mawatte	& draai\\
{\fontspec{Sazanami Gothic}個体}	&ko·tai	& wissel van partner [persoon]\\
{\fontspec{Sazanami Gothic}?理解}	&mo·ri·kai	& herhaal de oefening\\
{\fontspec{Sazanami Gothic}反対}	&han·tai	& gebruik de andere kant\\
{\fontspec{Sazanami Gothic}相手?}	&ai·te kaite	& zoek een andere partner\\
{\fontspec{Sazanami Gothic}礼}	&rei	& groeten, buigen [respect]\\
{\fontspec{Sazanami Gothic}?礼}	&ritsu·rei	& staande buiging\\
{\fontspec{Sazanami Gothic}座礼}	&za·rei	& zittende buiging\\
\end{tabular}
\vspace{.5cm}
}}

\def\g?{{
\noindent\par\begin{tabular}{@{}p{2.75cm}p{5cm}p{8.25cm}@{}}
{\fontspec{Sazanami Gothic}\Large\bfseries ?}	&\Large{\bfseries{?}}	& \Large{titels}\\
	&& \\
{\fontspec{Sazanami Gothic}大先生}	&\=o·sen·sei	& grote leraar\\
{\fontspec{Sazanami Gothic}先生}	&sen·sei	& leraar, instructeur\\
{\fontspec{Sazanami Gothic}道主}	&d\=o·shu	& degene die de weg wijst\\
{\fontspec{Sazanami Gothic}師範}	&shi·han	& hoofdinstructeur (minimaal 6e dan)\\
{\fontspec{Sazanami Gothic}指導員}	&shi·do·in	& instructeur, shihan-in-opleiding (4e of 5e dan)\\
{\fontspec{Sazanami Gothic}副指導員}	&fuku·shi·do·in	& assistent instructeur (2e of 3e dan)\\
{\fontspec{Sazanami Gothic}先輩}	&sen·pai	& iemands senior\\
{\fontspec{Sazanami Gothic}後輩}	&k\=o·hai	& iemands junior\\
{\fontspec{Sazanami Gothic}無段者}	&mu·dan·sha	& persoon zonder graad\\
{\fontspec{Sazanami Gothic}有段者}	&y\=u·dan·sha	& persoon met graad\\
{\fontspec{Sazanami Gothic}道場長}	&d\=o·j\=o·cho	& d\=oj\=o-hoofd\\
{\fontspec{Sazanami Gothic}弟子}	&de·shi	& leerling\\
{\fontspec{Sazanami Gothic}内弟子}	&uchi·de·shi	& in d\=oj\=o wonende leerling, inwonende leerling\\
{\fontspec{Sazanami Gothic}外弟子}	&soto·de·shi	& buiten d\=oj\=o wonende leerling\\
\end{tabular}
\vspace{.5cm}
}}

\def\g?{{
\noindent\par\begin{tabular}{@{}p{2.75cm}p{5cm}p{8.25cm}@{}}
{\fontspec{Sazanami Gothic}\Large\bfseries ?}	&\Large{\bfseries{?}}	& \Large{voorwerpen}\\
	&& \\
{\fontspec{Sazanami Gothic}道場}	&d\=o·j\=o	& trainingsruimte (plaats van verlichting) [plaats van de weg]\\
{\fontspec{Sazanami Gothic}上座}	&kamiza	& altaar met portret van \=osensei (eventueel met calligrafie of bloemen) [hoge kant van de mat]\\
{\fontspec{Sazanami Gothic}?座}	&shomiza	&  [(lage) achterkant van de mat]\\
{\fontspec{Sazanami Gothic}?}	&joseki	& (hoge) rechterzijde van de mat\\
{\fontspec{Sazanami Gothic}?}	&shimoseki	& (lage) linkerzijde van de mat\\
{\fontspec{Sazanami Gothic}畳}	&tatami	& matten\\
{\fontspec{Sazanami Gothic}本部}	&hom·bu	& hoofdkwartier\\
\end{tabular}
\vspace{.5cm}
}}

\def\gifuku{{
\noindent\par\begin{tabular}{@{}p{2.75cm}p{5cm}p{8.25cm}@{}}
{\fontspec{Sazanami Gothic}\Large\bfseries 衣服}	&\Large{\bfseries{i·fuku}}	& \Large{kleding}\\
	&& \\
{\fontspec{Sazanami Gothic}稽古着}	&kei·ko·gi	& oefenkleding, trainingspak\\
{\fontspec{Sazanami Gothic}帯}	&obi	& band\\
{\fontspec{Sazanami Gothic}白帯}	&shiro·obi	& witte band\\
{\fontspec{Sazanami Gothic}黑帯}	&kuro·obi	& zwarte band\\
{\fontspec{Sazanami Gothic}足袋}	&tabi	& soort van sokken met apparte grote teen\\
{\fontspec{Sazanami Gothic}草履}	&zori	& slippers\\
{\fontspec{Sazanami Gothic}袴}	&hakama	& traditionele Japanse broek\\
\end{tabular}
\vspace{.5cm}
}}

\def\gkakobogaku{{
\noindent\par\begin{tabular}{@{}p{2.75cm}p{5cm}p{8.25cm}@{}}
{\fontspec{Sazanami Gothic}\Large\bfseries 解剖学}	&\Large{\bfseries{kako·b\=o·gaku}}	& \Large{anatomie}\\
	&& \\
{\fontspec{Sazanami Gothic}腹}	&hara	& centrum, buik\\
{\fontspec{Sazanami Gothic}体}	&tai	& lichaam\\
{\fontspec{Sazanami Gothic}正面}	&sh\=o·men	& voorkant hoofd\\
{\fontspec{Sazanami Gothic}横面}	&yoko·men	& zijkand hoofd\\
{\fontspec{Sazanami Gothic}膝}	&hiza	& knie\\
{\fontspec{Sazanami Gothic}首}	&kubi	& nek\\
{\fontspec{Sazanami Gothic}胸}	&mune	& borst\\
{\fontspec{Sazanami Gothic}肩}	&kata	& schouder\\
{\fontspec{Sazanami Gothic}肘}	&hiji	& elleboog\\
{\fontspec{Sazanami Gothic}腕}	&ude	& arm\\
{\fontspec{Sazanami Gothic}手首}	&te·kubi	& pols\\
{\fontspec{Sazanami Gothic}手}	&te	& hand\\
{\fontspec{Sazanami Gothic}手刀}	&te·gatana	& handzwaard\\
{\fontspec{Sazanami Gothic}足}	&ashi	& been, voet\\
{\fontspec{Sazanami Gothic}足首}	&asho kubi	& enkel\\
{\fontspec{Sazanami Gothic}腰}	&koshi	& heupen, onderrug\\
{\fontspec{Sazanami Gothic}襟}	&eri	& kraag\\
{\fontspec{Sazanami Gothic}身}	&mi	& lichaam\\
{\fontspec{Sazanami Gothic}袖}	&sode	& mouw\\
\end{tabular}
\vspace{.5cm}
}}

\def\g?{{
\noindent\par\begin{tabular}{@{}p{2.75cm}p{5cm}p{8.25cm}@{}}
{\fontspec{Sazanami Gothic}\Large\bfseries ?}	&\Large{\bfseries{?}}	& \Large{diversen}\\
	&& \\
{\fontspec{Sazanami Gothic}合抜け}	&ai·nuke	& gezamelijk ontsnappen\\
{\fontspec{Sazanami Gothic}合打ち}	&ai·uchi	& elkaar tegelijkertijd doden\\
{\fontspec{Sazanami Gothic}当て身}	&a·te·mi	& ?\\
{\fontspec{Sazanami Gothic}武道}	&budo	& martiale weg\\
{\fontspec{Sazanami Gothic}組杖}	&kumi·j\=o	& stafoefeneningen met partner\\
{\fontspec{Sazanami Gothic}組太刀}	&kumi·ta·chi	& zwaardoefeneningen met partner\\
{\fontspec{Sazanami Gothic}太刀取り}	&ta·chi·do·ri	& zwaard afpakken\\
{\fontspec{Sazanami Gothic}短刀取り}	&tan·to·do·ri	& mes afpakken\\
{\fontspec{Sazanami Gothic}正眼}	&sei·gan	& wijzend naar het oog\\
{\fontspec{Sazanami Gothic}構え}	&kama·e	& ?\\
{\fontspec{Sazanami Gothic}間合い}	&ma·a·i	& juiste dynamische afstand [interval]\\
{\fontspec{Sazanami Gothic}膝行}	&shikkou	& lopen op knieën\\
{\fontspec{Sazanami Gothic}太鼓}	&tai·ko	& Japanse trommel die begin en einde van training aangeeft\\
\end{tabular}
\vspace{.5cm}
}}

\def\g?{{
\noindent\par\begin{tabular}{@{}p{2.75cm}p{5cm}p{8.25cm}@{}}
{\fontspec{Sazanami Gothic}\Large\bfseries 方}	&\Large{\bfseries{?}}	& \Large{richting}\\
	&& \\
{\fontspec{Sazanami Gothic}左}	&hidari	& links\\
{\fontspec{Sazanami Gothic}右}	&migi	& rechts\\
{\fontspec{Sazanami Gothic}入身}	&iri·mi	& ingaan\\
{\fontspec{Sazanami Gothic}回転}	&kai·ten	& wegdraaiend\\
{\fontspec{Sazanami Gothic}前}	&mae	& voorwaards\\
{\fontspec{Sazanami Gothic}後ろ}	&ushi·ro	& achterwaards\\
{\fontspec{Sazanami Gothic}横}	&yoko	& zijwaards\\
{\fontspec{Sazanami Gothic}内}	&uchi	& binnen, inwaarts\\
{\fontspec{Sazanami Gothic}外}	&soto	& buiten, buitenwaarts\\
{\fontspec{Sazanami Gothic}斜め}	&nana·me	& diagonaal\\
{\fontspec{Sazanami Gothic}直立}	&choku·ritsu	&  [verticaal]\\
{\fontspec{Sazanami Gothic}水平}	&sui·hei	&  [horizontaal]\\
{\fontspec{Sazanami Gothic}立て}	&ta·te	& staand\\
{\fontspec{Sazanami Gothic}反対}	&han·tai	& tegenover\\
{\fontspec{Sazanami Gothic}八方}	&hap·p\=o	& alle richtingen [acht richtingen]\\
\end{tabular}
\vspace{.5cm}
}}

\def\gundo{{
\noindent\par\begin{tabular}{@{}p{2.75cm}p{5cm}p{8.25cm}@{}}
{\fontspec{Sazanami Gothic}\Large\bfseries 運動}	&\Large{\bfseries{un·d\=o}}	& \Large{oefening}\\
	&& \\
{\fontspec{Sazanami Gothic}一教運動}	&ik·kyo undo	& ikky\=o oefening\\
{\fontspec{Sazanami Gothic}体の変更}	&tai no hen·kou	& basis lichaamsmenging\\
\end{tabular}
\vspace{.5cm}
}}

\def\ghojonokata{{
\noindent\par\begin{tabular}{@{}p{2.75cm}p{5cm}p{8.25cm}@{}}
{\fontspec{Sazanami Gothic}\Large\bfseries 法定之形}	&\Large{\bfseries{h\=o·j\=o no kata}}	& \Large{fundamentele principes}\\
	&& \\
{\fontspec{Sazanami Gothic}春の太刀}	&haru no ta·chi	&  [lente zwaard]\\
{\fontspec{Sazanami Gothic}夏の太刀}	&natsu no ta·chi	&  [zomer zwaard]\\
{\fontspec{Sazanami Gothic}秋の太刀}	&aki no ta·chi	&  [herfst zwaard]\\
{\fontspec{Sazanami Gothic}冬の太刀}	&fuyu no ta·chi	&  [winter zwaard]\\
{\fontspec{Sazanami Gothic}八相発破}	&hass\=o happa	& acht richtingen\\
{\fontspec{Sazanami Gothic}一刀両断}	&itto ry\=o·dan	& houw je ego doormidden\\
{\fontspec{Sazanami Gothic}右転左転}	&u·ten sa·ten	& veranderlijke tijden\\
{\fontspec{Sazanami Gothic}長短一身}	&cho·tan ichi·mi	& lang en kort zijn één\\
{\fontspec{Sazanami Gothic}受ける }	&u·ke·ru	&  [ontvangen]\\
{\fontspec{Sazanami Gothic}体剣}	&tai·ken	&  [lichaam zwaard]\\
{\fontspec{Sazanami Gothic}目礼}	&moku·rei	& knikkende groet zonder oogcontact te verbreken of te spreken\\
{\fontspec{Sazanami Gothic}抜剣}	&bak·ken	&  [eigen zwaard trekken]\\
{\fontspec{Sazanami Gothic}合?}	&ai seigan	& harmonieus wijzend naar het oog\\
{\fontspec{Sazanami Gothic}促進}	&soku·shin	&  [versnellen, haasten]\\
{\fontspec{Sazanami Gothic}仁王太刀}	&ni·\=o da·chi	&  [tempelbeschermer zwaard]\\
{\fontspec{Sazanami Gothic}?}	&kazashi	&  [schaduw]\\
{\fontspec{Sazanami Gothic}合上段}	&ai j\=o·dan	&  [harmonieuze j\=odan]\\
{\fontspec{Sazanami Gothic}?}	&tsume	&  [?]\\
{\fontspec{Sazanami Gothic}打ち込み}	&u·chi ko·mi	& herhaaldelijk oefenen [binnen ontelbaar]\\
{\fontspec{Sazanami Gothic}両腕}	&mor\=o·de	& beide armen\\
{\fontspec{Sazanami Gothic}押し込み}	&o·shi ko·mi	& duw oefening\\
{\fontspec{Sazanami Gothic}一文字}	&ichi·mon·ji	& rechte lijn\\
{\fontspec{Sazanami Gothic}?}	&so tai	&  [blokkeer lichaam]\\
{\fontspec{Sazanami Gothic}体当り}	&tai ata·ri	& bodycheck [lichaam stoot]\\
{\fontspec{Sazanami Gothic}?}	&hi tachi	& niet slaan\\
{\fontspec{Sazanami Gothic}霞}	&kasumi	& slaap [mist]\\
{\fontspec{Sazanami Gothic}一重み}	&hi·toe·mi	& lichaam zijwaarts naar opponent gedraaid\\
{\fontspec{Sazanami Gothic}?}	&muki	&  [aankijken, hoofd draaien]\\
{\fontspec{Sazanami Gothic}切り}	&ki·ri	&  [snijden]\\
{\fontspec{Sazanami Gothic}止め}	&to·me	&  [stoppen]\\
{\fontspec{Sazanami Gothic}裏打ち}	&ura u·chi	&  [rug slaan]\\
{\fontspec{Sazanami Gothic}曙光}	&sho·k\=o	&  [ochtendgloren, dageraad]\\
{\fontspec{Sazanami Gothic}大}	&dai	&  [groot]\\
{\fontspec{Sazanami Gothic}退けん}	&no·ke·n	&  [wegdoen]\\
\end{tabular}
\vspace{.5cm}
}}

\def\ggenkikai{{
\noindent\par\begin{tabular}{@{}p{2.75cm}p{5cm}p{8.25cm}@{}}
{\fontspec{Sazanami Gothic}\Large\bfseries 元気会}	&\Large{\bfseries{gen·ki·kai}}	& \Large{groep van gezondheisoefeningen}\\
	&& \\
{\fontspec{Sazanami Gothic}大円呼吸法}	&dai en ko·ky\=u h\=o	& ademhaling in grote cirkels [grote cirkel ademhalingsoefeningen]\\
{\fontspec{Sazanami Gothic}守有の呼吸}	&su·u no ko·ky\=u	&  [?]\\
{\fontspec{Sazanami Gothic}陽の手呼吸}	&yo no te ko·ky\=u	& ademhaling met handen in yang [?]\\
{\fontspec{Sazanami Gothic}陰の手呼吸}	&in no te ko·ky\=u	& ademhaling met handen in ying [?]\\
{\fontspec{Sazanami Gothic}気結びの手呼吸}	&ki·musu·bi no te ko·ky\=u	& ademhaling van de energie van handen die een kruis maken [energie menging ademhaling]\\
{\fontspec{Sazanami Gothic}阿吽の呼吸}	&a·un no ko·ky\=u	& ademhaling 'om één te worden met het omniversum' [aum ademhaling]\\
{\fontspec{Sazanami Gothic}元の呼吸}	&gen no ko·ky\=u	& breathing with principles\\
{\fontspec{Sazanami Gothic}寝運動}	&ne un·d\=o	&  [liggende oefeningen]\\
{\fontspec{Sazanami Gothic}揺動法}	&y\=o d\=o·h\=o	&  [oscillatiemethode]\\
{\fontspec{Sazanami Gothic}毛管運動}	&m\=o·kan un·d\=o	&  [?]\\
{\fontspec{Sazanami Gothic}合掌合蹠運動}	&gas·sh\=o gas·seki un·d\=o	&  [?]\\
{\fontspec{Sazanami Gothic}金魚運動}	&kin·gyo un·d\=o	&  [goudvis oefening]\\
{\fontspec{Sazanami Gothic}馬運動}	&uma un·d\=o	&  [paard oefening]\\
\end{tabular}
\vspace{.5cm}
}}

